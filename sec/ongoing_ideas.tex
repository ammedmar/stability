\section{Notes (Excluded from Final Paper)}

% \subsection{$\VR(\cp^n)$}

% \cite[Corollary 7.10]{lim2020vietoris} states that there exists some $\alpha_n\in (0,\arccos(-1/3))$ such that $\VR_r(\cp^n)$ is homotopy equivalent to $\cp^n$ for any $r\in (0,\alpha_n)$. 
% This is not enough for us, because we want a `long' enough bar in the usual barcode (assuming the Steenrod barcode is almost trivial). 

% If we consider the rational coefficients, then the rational filling radius of $\cp^n$ being $1/2\arccos(-1/3)$ can be helpful to us. %reference: \url{https://www.sciencedirect.com/science/article/pii/0166864191901223}
% The proposition in \cref{ss:filling_radius} gives the fundamental bar  $(0,\arccos(-1/3))\in \Hbarc[\mathbb{Q}]{\degp}{\cM}$.

% The next question will be what are the Steenrod barcodes (over $\mathbb{Q}$) of $\cp^n$. This can still be a hard question, because we do not know the homotopy type of $\VR_r(\cp^n)$ for $r$ in the fundamental bar.

%\ling{to-do's for Ling: reorganize Section 5.5. Mention the gaps we have, e.g. what is lacking if we were to apply the same approach ($\VR(S^n/\sim)\simeq \VR(S^n)/\sim$ is not that difficult, but $\VR(S^n)/\sim$ is hard to understand)}

%We estimate the bottleneck distance $\db$ between the Steenrod barcodes of the two space $\VS^n$ and $\rp^n$, and show that it provides a better (lower-bound) approximation of the Gromov--Hausdorff distance than $\db$ between the standard barcodes.

\subsection{First critical value of $\opH_\degp(\VR(\rp^n))$}

In this section, we show that for any degree $1 \leq \degp \leq n$, the first critical value of the $\degp$-th homology of the Vietoris-Rips filtration of $\rp^n$ is $\tfrac{2\pi}{3}.$

We fix some $n\geq 2.$

\subsubsection{}

\ling{We need to be careful with the citation here: (1) cite \cite{adams2022metric} for the original definition; (2) cite Liam Barham for the fixed definition. 
We should ask Liam Barham for permission to cite his unpublished work.}

Let $G$ be a group acting properly and by isometries on a metric space $X$.
Let $t_0>0$. The action of a group $G$ on $X$ is an \defn{$t_0^*$-diameter action} if for any non-negative integer $k$, $\diam_{X_G}\{[x_0],\dots,[x_k]\}<t_0$ implies that there exist unique choice of $g_i$'s for $1\leq i\leq k$ such that $\diam_{X}\{x_0,g_1x_1,\dots,g_kx_k\}<t_0$. 

Assume the action of $G$ on $X$ is a $t_0^*$-diameter action for some $t_0> 0$.
In \cite[Proposition 3.5]{adams2022metric}, the authors constructed a homomeorphism from $\VR_t^m(X_G)$ to $\VR_t^m(X)/G$ for all $t \leq t_0$.
We review their constructions below. 

Define the map $h$ to be
\[
h \colon \VR_t^m(X) \to \VR_t^m(X_G) 
\text{ with }
\sum_{i=1}^k \lambda_i x_i \mapsto \sum_{i=1}^k \lambda_i [x_i],
\]
Because $G$ acts isometrically, $h$ is well-defined.
Because two points in the same orbit of the $G$ action always have the same image under $h$, it induces a map $h/G \colon \VR_t^m(X)/G \to \VR_t^m(X_G)$.
Moreover, $h/G$ is an isomorphism, following from the fact that the action of $G$ on $X$ is an $r^*$-diameter action; see \cite[Proposition 3.5]{adams2022metric} for further details.

Assuming that the action of $G$ on $X$ is a $t_0^*$-diameter action, for any $0<t<t_0$, we can consider the inverse of $h/G$ and we denote it as
\[
h \colon \VR_t^m(X_G) \to \VR_t^m(X)/G
\text{ with }
\sum_{i=1}^k \lambda_i [x_i] \mapsto [\sum_{i=1}^k \lambda_i x_i].
\]


\subsubsection{}

Let $\bS^n$ the sphere of radius $2$ and let $G=\Z/2\Z$ act on $\bS^n$ by identifying the antipodal points.
It follows from \cite[Corollary]{adams2022metric} that the action of $G$ on $\bS^n$ is a $t^*$-diameter action for any $t<\tfrac{2\pi}{3}$.

For any $t<\pi$, let $\tilde{f}^n$ be the composition
\[
    \tilde{f}^n \colon \VR_t^m(\bS^n) \to \R^{n+1} \setminus \set{0} \xra{\pi^n} \bS^n,
\]
where the first map sends a formal linear sum $\sum_{i=1}^k \lambda_i x_i$ in the Vietoris--Rips thickening $\VR_t^m(\bS^n)$ to the point $\sum_{i=1}^k \lambda_i x_i \in \bbR^{n+1}$ where $x_i \in \bS^n$ and $\lambda_i \in [0,1]$ satisfying $\sum_i\lambda_i=1$, and the second map $\pi^n$ is the radial projection map.

By \cite[Proposition 5.3]{adamaszek2018metric}, the map $\tilde{f}^n$ is a homotopy equivalence for all $0<t<\zeta_n=\arccos{(-\tfrac{1}{n+1})}$.
% When $n=1$, this is $\tfrac{4\pi}{3}.$

\ling{To Anibal: please check that $\tilde{f}^n/G$ is well-defined for lens spaces and if it is a homotopy equivalence. }

Because $\tilde{f}^n$ preserves group action, i.e. $\tilde{f}^n(x) = \tilde{f}^n(gx)$ for any $x\in \bS^n$ and $g\in G$, we have the homotopy equivalence for $0<t<\zeta_n$, %\anibal{Maybe use superscripts for these maps and the $\rho$ too?}
\[
\tilde{f}^n/G \colon 
\VR_t^m(\bS^n)/G \, 
\to \bS^n_G.
%\cong \rp^n.
\]
By composing $h^n$ with $\tilde{f}^n/G$, we obtain a map 
\[
\rho^n = h^n \circ \tilde{f}^n/G 
\colon \VR_t^m(\bS^n_G) \to \bS^n_G
\]

\subsubsection{} 

For $0<t\leq s \leq s'$, we consider the following diagram of topological spaces:
\begin{equation}\label{d:fundamental_bars_diagram}
    \begin{tikzcd}
        \bS^{n-1}_G
        \ar[d, hook,"{\iota}" left]
        &
        \VR_t(\bS^{n-1}_G)
        \ar[d, hook,"\iota_t"]
        \ar[l, "\rho^{n-1}" above, "\simeq" below]
        \ar[r, hook, "v^{n-1}"]
        &
        \VR_{s}(\bS^{n-1}_G)
        \ar[d, hook]
        \\
        \bS^{n}_G
        &
        \VR_t(\bS^{n}_G)
        \ar[l, "\rho^n" below, "\simeq" above]
        \ar[r, hook, "v^{n}"]
        &
        \VR_{s'}(\bS^{n}_G).
    \end{tikzcd}
\end{equation}
Here, the horizontal inclusions $v^{n-1}$ and $v^n$ are induced by the Vietoris--Rips filtration, whereas the vertical maps are induced by the equatorial inclusion of real projective spaces $\iota \colon \bS^{n-1}_G \hookrightarrow \bS^{n}_G$.

We claim that diagram \eqref{d:fundamental_bars_diagram} commutes. 
The commutativity of the right-hand-side square is straightforward.
For the left-hand-side square, we need to verify that $\iota \circ \rho^{n-1}=\rho^{n} \circ \iota_t$.
\ling{notation for maps below need to be fixed at the end.}
Indeed, for any $y = \sum_{i=1}^k \lambda_i [x_i] \in \VR_t(\bS^{n-1}_G)$, we have
\begin{center}
    $(\iota \circ \rho^{n-1})(y)
    =\iota(f^{n-1}/\sim([\sum_i \lambda_i x_i]))
    =\iota([f^{n-1}(\sum_i \lambda_i x_i)])
    =[\pi^{n-1}(\sum_i \lambda_i x_i)]
    $
\end{center}
as an element in $\bS^n_G$, and
\begin{center}
    $(\rho^{n} \circ \iota_t)(y) = \rho^{n}(y) = f^{n}/\sim([\sum_i \lambda_i x_i]) = [f^{n}(\sum_{i=1}^k \lambda_i x_i)] = [\pi^{n}(\sum_{i=1}^k \lambda_i x_i)]
    $
\end{center}
Because $\pi^{n}$ restricted to $\bS^{n-1}_G$ is equal to $\pi^{n-1}$, we conclude that $(\iota \circ \rho^{n-1})(y) = (\rho^n \circ \iota_t)(y)$ for any $y$.
Thus, the claim holds.

\subsubsection{}

\lemma 
\ling{Here mention the several values; make assumptions.}
For integers $1 \leq \degp \leq n$,
\[
\left(0, \tfrac{2\pi}{3}\right) \in \Hbarc{\degp}{\bS^n_G}.
\]

\ling{We need the first critical value of the $\degp$-th homology of $\bS^n_G$ for all $1\leq \degp\leq n-1$ and all $n.$}

\begin{proof}
	We will use an induction argument on $n$.
	When $n = 1$, \cref{ss:filling_radius} implies that
 \ling{here we need the filling radius of $\bS^1/G$}
	\[
	(0, 2\fillrad{\rp^1}) = \left(0, \tfrac{2\pi}{3}\right) \in \Hbarc{1}{\rp^1}.
	\]
	Assume the statement holds for $\bS^{n-1}_G$.
	% That is, for any $1 \leq \degp \leq n-1$,
	% \[
	% \left(0, \tfrac{2\pi}{3}\right) \in \Hbarc{\degp}{\bS^{n-1}_G}.
	% \]

    For the case when $1\leq \degp\leq n-1$, applying the $\degp$-th homology functor (over $\Ftwo$ \ling{replace with a suitable field}) to diagram \eqref{d:fundamental_bars_diagram}, we obtain the following commutative diagram of vector spaces:
    \ling{For the first $\frac{2\pi}{3}$, we want to replace it with a value such that $\rH_\degp(v^{n-1})$ is zero. 
    The first critical value may not be enough (unless the homology is always at most dimension one): changing type is different from being zero.}
    \[
    \begin{tikzcd}
        \rH_\degp(\bS^{n-1}_G)
        \ar[d, "\cong" left]
        &
        \rH_\degp(\VR_t(\bS^{n-1}_G))
        \ar[d, "\rH_\degp(\iota_t)" left, "\cong" right, myred]
        \ar[l, "\cong" above]
        \ar[rr, "\rH_\degp(v^{n-1})=0", myred]
        &
        &
        \rH_\degp(\VR_{\tfrac{2\pi}{3}+\epsilon}(\bS^{n-1}_G))
        \ar[d]
        \\
        \rH_\degp(\bS^{n}_G)
        &
        \rH_\degp(\VR_t(\bS^{n}_G))
        \ar[l, "\cong"]
        \ar[rr, "\rH_\degp(v^n)" , myred]
        &
        &
        \rH_\degp(\VR_{\tfrac{2\pi}{3}+\epsilon}(\bS^{n}_G)).
    \end{tikzcd}
    \]
    Here, $\rH_\degp(g)$ for some map $g$ between topological spaces denotes the induced map of $g$ on the $\degp$-th homology. 

    Commutativity of the left-hand-side square implies that $\rH_\degp(\iota_t)$ is an isomorphism.
    By induction assumption, $(0, \tfrac{2\pi}{3})\in\Hbarc{\degp}{\bS^{n-1}_G}$, which implies that $\rH_\degp(v^{n-1})$ is the zero map.
    Then, using the right-hand-side square's commutativity, we deduce $\rH_\degp(v^n) \circ \rH_\degp(\iota_t)=0$.
    Since $\rH_\degp(\iota_t)$ is an isomorphism, we conclude that $\rH_\degp(v^n)=0$.
    This holds for any $\epsilon>0$, so we can conclude $(0, \tfrac{2\pi}{3})\in\Hbarc{\degp}{\bS^{n}_G}$.
	
	For the case when $\degp=n$, we can apply the lemma in \textsection \ref{ss:filling_radius} to obtain 
    \[
    (0,2\fillrad{\bS^n_G}) = (0, \tfrac{2\pi}{3}) \in \Hbarc{n}{\bS^n_G}.
    \]
        
	Therefore, $(0, \tfrac{2\pi}{3}) \in \Hbarc{\degp}{\bS^n_G}$ is established for all $1\leq \degp\leq n.$
\end{proof}

We will refer to the bar in this lemma as the \defn{fundamental bar} of $\bS^n_G$.
