% !TEX root = ../vr_st.tex

\section*{Related work}

\subsection*{Decomposition formulas}

Both \cref{thm:decomposition1} and \cref{thm:decomposition2} build on previous studies of the Vietoris--Rips filtration of metric spaces.
Specifically of their wedge sums \cite{adamaszek2020homotopy} and products \cite{adamaszek2017VietorisProduct, gakhar2019k, lim2024vietoris}.

\subsection*{Stability}

In \cite[Theorem~87]{ginot2019distances}, Ginot and Leray prove a Gromov--Hausdorff inequality using the \textit{Steenrod interleaving distance}.
This is the interleaving distance in the category of persistent algebras that are also persistent modules over the Steenrod algebra \(\mathcal{A}_p\).
Our stability result, \cref{thm:stability intro}, holds for all cohomology operations.
In the important case of Steenrod operations, it seems likely that our result could be deduced from theirs.

Our proof strategy builds on the techniques developed in \cite{zhou2023beyond,memoli2024persistent}.

\subsection*{Gromov--Hausdorff estimates via interleaving estimates}

The exact value of the Gromov--Hausdorff distance is known in only a small number of cases \cite{memoli2012some,ji2021gromov,adams2022gromov,talipov2022gromov,lim2021gromov,harrison2023quantitative,saul2024gromov,saul2024some}.
Much of the effort in these works, as well as in the broader study of the Gromov--Hausdorff distance, has been devoted to establishing effective bounds on this distance (see \cite{lim2021gromov} for a comprehensive discussion of the literature).

In this paper, we contribute to the line of work about estimating Gromov--Hausdorff distances between metric spaces by seeking lower bounds based on persistent cohomology operations.
Indeed, \cref{thm:inequalities intro} illustrates the stronger discriminating power of persistent cohomology operations over (standard) persistent homology.
It does so by providing bounds on the interleaving distance of the persistent homology and persistent cohomology operations of the Vietoris--Rips filtration of pairs of Riemannian pseudo-manifolds.
For example, combining \cref{thm:stability intro} and \cref{thm:inequalities intro}, we establish a lower bound of $\frac{\pi}{6}$ for the Gromov--Hausdorff distances between \(\rp^n\) and \(\bS_{\rp^n}\), which is strictly greater than the lower bound obtained from persistent homology, which does not exceed $\frac{\pi}{8}$.
See \cref{s:gh_estimates} for more details.

This approach, which employs topological invariants more informative than persistent homology, builds upon prior work explored in \cite{zhou2023beyond, zhou2023persistent, memoli2024persistenthomotopy, memoli2024persistent, memoli2025ephemeral}.
These studies have investigated sharper lower bounds on the Gromov--Hausdorff distance by leveraging persistent invariants derived from homotopy groups, rational homotopy groups, the cohomology ring, the Lyusternik--Schnirelmann category, Sullivan minimal models, and filtered chain complexes.

\subsection*{Critical radii}

The Kuratowski embedding \(K \colon \cX \to \rL^\infty(\cX)\) of a metric space \(\cX\) is its canonical isometric inclusion, given by \(\cX \ni x \mapsto d_\cX(x,\cdot)\), into the Banach space $\rL^\infty(\cX)$.
For any $r>0$, let \(\rU_r(\cX)\) denote the radius \(r\) neighborhood of \(K(\cX)\) in \(\rL^\infty(\cX)\).
It was established in \cite{chazal2009gromov} for finite metric spaces and later extended to compact metric spaces in \cite{lim2024vietoris} that the Vietoris--Rips complex \(\VR_{2r}(\cX)\) is naturally homotopy equivalent to \(\rU_r(\cX)\) for every \(r > 0\).

Gromov's filling radius is defined as the smallest \(r\) for which the fundamental class of a closed connected Riemannian manifold becomes null-homologous in \(\rU_r(\cM)\).
The use of this invariant and other similar critical radii is crucial in our proof of \cref{thm:inequalities intro}.
This approach is similar to the one used in \cite[Prop.~9.38]{lim2024vietoris} where the authors consider pairs of spaces consisting of an $n$-manifold \(\cM\) and a space \(\cX\) with trivial \(n\)-homology, and then bound the interleaving distance of the \(n\)-th persistent homology by the filling radius of \(\cM\).
In contrast, we consider a space \(\cX\) whose homology in all degrees matches that of \(\cM\).