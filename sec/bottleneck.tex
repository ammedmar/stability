% !TEX root = ../vr_st.tex

\section{Gromov--Hausdorff estimates}\label{s:gh_estimates}

In this section, we construct examples of Riemannian pseudomanifolds where Gromov--Hausdorff estimates from persistent cohomology operations yield tighter bounds than those from persistent homology.
This demonstrates the enhanced discriminating power of these persistent invariants in Riemannian geometry.

\subsection{General estimates}\label{ss:genberal_distance_comparison}

\subsubsection{}

We will prove the relevant statements in greater generality, allowing for similar estimates to be established for Lens spaces once their critical radii are determined determined.

Let $\cM$ be a closed connected Riemannian \(n\)-manifold of diameter $\pi$.

\medskip\noindent\textit{\darkblue Desiderata}.\
\begin{enumerate}
	\item $\crit(\cM) \geq \tfrac{\zeta_n}{2} $;
	\item $\firstdeath{m}{\cM} < \tfrac{\zeta_n}{4}+\tfrac{\zeta_m}{2}$ for any $m \in \N$.
	\item There exists $\theta \in \cO(\ell, m)$ with \(\ell \neq m\) such that $\img_\theta(\cM) \neq 0$.
\end{enumerate}

\medskip As shown in \cref{s:critical_radii_rpn}, round real projective spaces of radius \(\pi\) satisfy these Desiderata.
For fixed $n\in \N$ and the round Lens space $\rL^n_p$ of radius \(\pi\), Desideratum~(3) is achieved by selecting $\theta$ as an appropriate Steenrod reduced power (see \cref{sss:cohomology_lens}).
Desideratum~(1) remains unproven, but for Desideratum~(2) it suffices to check that
\[
\fillrad(\rL^1_p) \leq \dotsb \leq \fillrad(\rL^n_p) < \tfrac{3\zeta_{2n+1}}{4}.
\]
Because if the above condition holds, then \cref{s:critical_radii_lens} implies that for any degree $m \leq 2n+1$,
\[
\firstdeath{m}{\rL^n_p} \leq \fillrad(\rL^n_p) < \tfrac{3\zeta_{2n+1}}{4} \leq \tfrac{\zeta_{2n+1}}{4} + \tfrac{\zeta_{m}}{2}.
\]

\subsubsection{}\label{sss:choice of S_M}

Let \(\cM\) be a closed connected Riemannian \(n\)-manifold of radius \(\pi\) satisfying our Desiderata.
Let $\bS_{\cM}$ be the wedge sum of spheres of radius \(\pi\) with the same reduced homology groups as $\cM$.

\theorem
There is $\ell, m \in \N$ and $\theta \in \cO(\ell, m)$ such that
\begin{equation}\label{eq:comparison}
	\di \Big(\rH^{\VR}_{m}(\bS_{\cM}),\, \rH^{\VR}_{m}(\cM)\Big) < \di \big(\img^{\VR}_{\theta}(\bS_{\cM}),\, \img^{\VR}_{\theta}(\cM)\big).
\end{equation}

The proof of this theorem involves the following three steps, which we will respectively prove in \cref{sss:db_upper_bound}, \cref{sss:db_theta_lower_bound}, and \cref{sss:comparison_lemma_zeta_n}.
\begin{enumerate}
	\item [(a)] LHS of \cref{eq:comparison} is bounded above by $\max\big\{|\zeta_m - 2\firstdeath{m}{\cM} |, \tfrac{\pi - \zeta_n}{2}\big\}$.
	\item [(b)] RHS of \cref{eq:comparison} is bounded below by $\min\{\firstdeath{\theta}{\cM}, \zeta_n\}$.
	\item [(c)] The value in (a) is strictly less than the value in (b).
\end{enumerate}

\subsubsection{}\label{sss:db_upper_bound}

\lemma
If \(\opH_m(\cM)\) is non-zero, then
\begin{equation}\label{eq:db_usual_upper_bound}
	\di\Big(\rH^{\VR}_{m}(\bS_{\cM}),\, \rH^{\VR}_{m}(\cM)\Big)
	\leq \max\big\{|\zeta_m - 2\firstdeath{m}{\cM} |,\, \tfrac{\pi - \zeta_n}{2}\big\}.
\end{equation}
If \(\opH_m(\cM)\) is zero, a similar result holds, with the upper bound being $\tfrac{\pi - \zeta_n}{2}$.

\begin{proof}
	Because both metric spaces are totally bounded, their persistent homology is q-tame.
	By \cref{ss:algebraic_stability}, the interleaving distance between their persistent homology equals the bottleneck distance between the corresponding barcodes.

	To obtain an upper bound on the bottleneck distance, it suffices to construct a matching and calculate its cost.
	Let \(\beta_m = 2\firstdeath{m}{\cM}\).
	First, suppose \(\opH_m(\cM)\) is non-zero.
	Consider the matching \(P\) between \(\Hbarc[\field]{m}{\bS_{\cM}}\) and \(\Hbarc[\field]{m}{\cM}\) such that the bar \((0, \zeta_m)\) from \(\Hbarc[\field]{m}{\bS_{\cM}}\) is matched to \((0, \beta_m)\) from \(\Hbarc[\field]{m}{\cM}\), with all other bars remaining unmatched. Using the estimates from \cref{fig:barcodes_general} and \cref{fig:barcodes_vs}, we find that the cost of \(P\) satisfies
	\begin{align*}
		\cost(P)
		= & \max\Big\{|\zeta_m - \beta_m|,\, \frac{\pi - \zeta_n}{2},\, \frac{\pi - 2\crit(\cM)}{2}\Big\} \\
		= & \max\Big\{|\zeta_m - \beta_m|,\, \frac{\pi - \zeta_n}{2}\Big\} \quad \text{(since \(\zeta_n \leq 2\crit(\cM)\))}.
	\end{align*}
	Thus, \cref{eq:db_usual_upper_bound} follows.

	Now, suppose \(\opH_m(\cM)\) is zero. In this case, the matching where all bars are unmatched has a cost no larger than \(\frac{\pi - \zeta_n}{2}\).
\end{proof}

\subsubsection{}\label{sss:db_theta_lower_bound}

\lemma
Let $\theta \in \cO(\ell, m)$ be such that $\img \theta_\cM$ is non-zero.
Then,
\begin{equation}\label{eq:db_theta_lower_bound}
	\di\big(\img^{\VR}_{\theta}(\bS_{\cM}), \img^{\VR}_{\theta}(\cM)\big)
	\geq \min\{\firstdeath{\theta}{\cM}, \zeta_n\}.
\end{equation}

\begin{proof}
	Since both metric spaces are totally bounded, their persistent cohomology is q-tame, making the persistent $\img_\theta$-modules q-tame as well (see \cref{ss:theta-modules-q-tame}).
	Thus, the left-hand-side of \cref{eq:db_theta_lower_bound} can be replaced with the bottleneck distance between the corresponding $\img_\theta$-barcodes.

	To establish a lower bound on this bottleneck distance, we need to compute the minimum cost of matching specific bars.

	Let \(\gamma_\theta = 2\firstdeath{\theta}{\cM}\).
	From \cref{fig:barcodes_general}, note that \(\thetabarc{\cM}\) contains a bar \((0, \gamma_\theta)\).
	We now calculate the minimum cost of matching this bar.

	Consider an arbitrary matching \(Q\) between \(\thetabarc{\bS_{\cM}}\) and \(\thetabarc{\cM}\).
	There are two cases to analyze.
	If $(0,\gamma_\theta )$ is unmatched in $Q$, then $\cost(Q)$ is at least $ \tfrac{\gamma_\theta }{2}$.
	If $(0,\gamma_\theta )$ is matched to some bar $(a,b) \in \thetabarc{\bS_{\cM}}$, then
	$\cost(Q) = \|(0,\gamma_\theta ) - (a,b)\|_\infty \geq a \geq \zeta_n$, because $(a,b) \subset (\zeta_n, \pi)$.
	Thus, any matching $Q$ must satisfy $\cost(Q) \geq \min\{\tfrac{\gamma_\theta }{2}, \zeta_n\}$, which proves \cref{eq:db_theta_lower_bound}.
\end{proof}

\subsubsection{}\label{sss:comparison_lemma}

The following results will be used in \cref{sss:comparison_lemma_zeta_n} and \cref{ss:distance_estimate_rpn}.

As before, let $\zeta_n = \arccos{(-\tfrac{1}{n+1})}$, which decreases as $n$ increases.
We derive the following lemma, which, while straightforward, requires careful exposition.

\lemma
Let $m \leq n$.
\begin{enumerate}
	\item \(\zeta_m - \zeta_n < \tfrac{\pi - \zeta_n}{2} < \tfrac{\zeta_n}{2}.\)
	\item For any $r \in \big[\zeta_n, \tfrac{\zeta_n}{2}+\zeta_m\big)$, $\max\big\{|\zeta_m - r |, \tfrac{\pi - \zeta_n}{2}\big\} < \tfrac{\zeta_n}{2}$.
\end{enumerate}

\begin{proof}
	(1) The first inequality follows from $\zeta_m - \zeta_n < \tfrac{\pi}{6} < \tfrac{\pi - \zeta_n}{2}$.
	The second inequality holds because $\zeta_n$ is larger than $\tfrac{\pi}{2}$.

	(2) Since $\tfrac{\pi - \zeta_n}{2}$ is less than $\tfrac{\zeta_n}{2}$, it remains to show $|\zeta_m - r |$ is also less than $\tfrac{\zeta_n}{2}$, which follows from
	\(
	|\zeta_m - r | < \max\big\{\zeta_m - \zeta_n, \tfrac{\zeta_n}{2}\big\} = \tfrac{\zeta_n}{2},
	\) by Part (1).
\end{proof}

\subsubsection{}\label{sss:comparison_lemma_zeta_n}

We will finish the proof of \cref{sss:choice of S_M} by completing the third step outlined therein.

\lemma
For any $m \in \N$ and $\theta \in \cO(\ell, m)$ such that $\img \theta_\cM$ is non-zero,
\[\max\big\{|\zeta_m - 2\firstdeath{m}{\cM} |, \tfrac{\pi - \zeta_n}{2}\big\} < \min\{\firstdeath{\theta}{\cM}, \zeta_n\}.\]
\begin{proof}
	Let $\beta_m = 2\firstdeath{m}{\cM}$ and $\gamma_\theta = 2\firstdeath{\theta}{\cM}$.
	Because \(\zeta_n \leq 2\crit(\cM) \leq \beta_m < \tfrac{\zeta_n}{2} + \zeta_m\), $\beta_m$ satisfies the condition in Part (2) of the above lemma.
	Thus, we have the leftmost inequality below:
	\[
	\max\big\{|\zeta_m - \beta_m|, \tfrac{\pi - \zeta_n}{2}\big\} < \tfrac{\zeta_n}{2}
	\leq
	\min\big\{\tfrac{\gamma_\theta}{2}, \zeta_n\big\}.
	\]
	To obtain the rightmost inequality, note that \(\zeta_n \leq 2\crit(\cM) \leq \gamma_\theta\).
\end{proof}

\subsection{Real projective spaces}\label{ss:distance_estimate_rpn}

As stated earlier, the round \(\rp^n\) of radius \(\pi\) satisfies our Desiderata, so the discussions in the previous subsection apply.
Here we obtain a more precise estimate of the interleaving distances involved.
Let \(\bS_{\rp^n} = \bS^1 \vee \dots \vee \bS^n\) as before.
Then, we have the following result.

\theorem
(1) For any \(n,m \in \N\),
\[
\di\Big(\rH^\VR_m(\bS_{\rp^n}),\, \rH^\VR_m(\rp^n)\Big) < \frac{\pi - \zeta_n}{2} < \frac{\pi}{4}.
\]

\noindent (2) For any \(n,m,k \in \N\) such that \(m \leq n\), \(k \leq \tfrac{n-1}{2}\), and \(\binom{m-k}{k}\) is odd:
\[
\di\Big(\img_{\Sq^k}^\VR(\bS_{\rp^n}),\, \img_{\Sq^k}^\VR(\rp^n)\Big) \geq \frac{\pi}{3},
\]
where $\Sq^k$ lands in degree \(k\).

\begin{proof}
	(1) When $1 \leq m \leq n$, we have shown in \cref{s:critical_radii_rpn} that
	$2\firstdeath{m}{\rp^n} = \tfrac{2\pi}{3}$.
	Thus, the upper bound we computed in \cref{eq:db_usual_upper_bound} on the interleaving distance between persistent homology evaluates as
	\[\max\big\{|\zeta_m - 2\firstdeath{m}{\cM} |, \tfrac{\pi - \zeta_n}{2}\big\} =
	\max \big\{ |\zeta_m - \tfrac{2\pi}{3} |, \tfrac{\pi - \zeta_n}{2} \big\}
	= \tfrac{\pi - \zeta_n}{2}.
	\]
	Here, we applied \cref{sss:comparison_lemma} to deduce that $|\zeta_m - \tfrac{2\pi}{3} | = \zeta_1 - \zeta_m < \tfrac{\pi - \zeta_1}{2} \leq \tfrac{\pi - \zeta_n}{2}$.

	(2) With $n,k,\Sq^k$ satisfying the required conditions, we have shown in \cref{s:critical_radii_rpn} that
	$\firstdeath{\Sq^k}{\rp^n} = \tfrac{2\pi}{3}$.
	Applying \cref{eq:db_theta_lower_bound}, we have
	\[\di\Big(\img_{\Sq^k}^\VR(\bS_{\rp^n}),\, \img_{\Sq^k}^\VR(\rp^n)\Big)
	\geq \min\Big\{\tfrac{\pi}{3},\, \zeta_n\Big\}
	= \tfrac{\pi}{3}.\qedhere
	\]
\end{proof}

By combining the above theorem with \cref{thm:stability intro}, we derive a lower bound of \(\frac{1}{2} \cdot \frac{\pi}{3}\) for the Gromov–Hausdorff distance between \(\rp^n\) and \(\bS_{\rp^n}\), larger than the bound obtained from persistent homology, which is at most \(\frac{1}{2} \cdot \frac{\pi}{4}\).
This result demonstrates the stronger distinguishing power of persistent cohomology operations compared to persistent homology.