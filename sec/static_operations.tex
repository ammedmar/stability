\subsection{Eilenberg--MacLane spaces}

\subsubsection{}

We denote by $K(\pi, n)$ any cellular space which has only one non-trivial homotopy group, namely, $\pi_n(K(\pi, n)) = \pi$.
Such space is unique up to homotopy, and is referred to as an \defn{Eilenberg--MacLane space}.
For example, since the circle has universal cover the real line and fundamental group isomorphic to $\Z$, it is a model for $K(\Z,1)$.

Let $\cX$ be a space.
The \defn{Hurewicz homomorphism} $h \colon \pi_i(\cX) \to \rH_i(\cX)$ is defined by choosing a generator $u$ of $\rH_i(\bS^i)$ and sending the homotopy class $[f]$ of a based map $f \colon \bS^i \to \cX$ to $f_*(u)$, where $\bS^i$ denotes the $i$-sphere.
Using the properties of this map and the universal coefficient theorem for cohomology one shows the existence of a preferred class $\iota_n \in \rH^n(K(\pi, n); \pi)$ termed the \defn{fundamental class} of $K(\pi, n)$.\footnote{The Hurewicz theorem states that if $\cX$ is such that $\pi_i(\cX)$ is trivial for all $i \leq n$, then the Hurewicz homomorphism is an isomorphism in dimensions $i \leq n$ and a surjection in dimension $n+1$.
	Hence, the first non-trivial homotopy group of $\cX$ and the first non-trivial homology group occur in the same dimension and are isomorphic under $h$.
	For $n=1$, the theorem requires certain modifications which do not affect the following analysis.
	The group $\Hom(\rH_n(\cX), \pi_n(\cX))$ contains $h^{-1}$, the inverse of the Hurewicz homomorphism.
	Subject to the hypothesis on $\cX$, the universal coefficient theorem for homology gives an isomorphism of $\Hom(\rH_n(\cX), \pi_n(\cX))$ and $\rH^n(\cX; \pi_n(\cX))$, and the fundamental class $\iota \in \rH^n(\cX; \pi_n(\cX))$ corresponds to $h^{-1}$ under this isomorphism.}

The relevance of Eilenberg--MacLane space for the study of cohomology is the following bijection holding for any space \(\cX\) with the homotopy type of a CW complex:
\[
\begin{tikzcd}[column sep=small, row sep=0]
	{[\cX, K(\pi, n)]} \rar & \rH^n(\cX; \pi) \\
	{[f]} \rar[maps to] & f^*(\iota_n)
\end{tikzcd}
\]
where $[\cX, K(\pi, n)]$ denotes the set of homotopy classes of based maps from $\cX$ to $K(\pi, n)$.
We refer to this bijection as the \defn{representability of cohomology}.

\subsection{Cohomology operations}

\subsubsection{}
Let both $\pi$ and $G$ be abelian groups.
A \defn{cohomology operation} of type $(\pi, n; G, m)$ is a family of functions
\[
\theta_\cX \colon \rH^n(\cX; \pi) \to \rH^m(\cX; G),
\]
one for each cellular space $\cX$, satisfying the naturality condition $f^* \theta_{\cY} = \theta_{\cX} f^*$ for any map $f \colon \cX \to \cY$.
We will denote by $\cO(\pi, n; G,m)$ the set of cohomology operations of type $(\pi, n; G,m)$.

\subsubsection{}
There is bijection between $\cO(\pi, n; G,m)$ and homotopy classes of based maps between their Eilenberg--MacLane spaces.
Explicitly, given $[f] \in [K(\pi,n), k(G,m)]$ and an $n$-cohomology class in a space $\cX$ with $\pi$ coefficients, say $g^*(\iota_m)$ for some $g \colon \cX \to K(G, m)$, the value of the cohomology operation $\theta_{\cX}^{[f]}$ on it is
\[
\theta_{\cX}^{[f]}(g^*(\iota_m)) = (f \circ g)^*(\iota_m).
\]
In other words, cohomology operations are parameterized by the cohomology of Eilenberg--MacLane spaces, i.e.,
\[
\cO(\pi, n; G,m) \cong \rH^m(K(\pi,n); G).
\]

\subsection{Steenrod operations}\label{ss:steenrod}

\subsubsection{} The \defn{suspension} of a space $\cX$, denoted as $\sus \cX$, is the product of $\cX$ with the interval $[0,1]$ having each $\cX \times \set{0}$ and $\cX \times \set{1}$ collapsed to a point.
For example, $\sus \bS^n$ is homotopy equivalent to $\bS^{n+1}$.

The inclusion of $\cX$ into $\sus \cX$ induces an isomorphism
\[
\rH^n(\cX) \cong \rH^{n+1}(\sus \cX)
\]
in cohomology with any coefficients, referred to as the \defn{suspension isomorphism}.

\subsubsection{} A \defn{stable cohomology operation of degree $k$} is a natural family of cohomology operations
%\[
%\set{\theta^n_\cX \colon \rH^n(\cX; \pi) \to \rH^{n+k}(\cX; G)}_{n \in \N}
%\]
\[
\set{\theta^n \in \cO(n,\pi,n+k,G)}_{n \in \N}
\]
commuting with suspension isomorphisms, i.e., making the diagram
\[
\begin{tikzcd}
	\rH^n(\cX; \pi) \rar["\theta^n_\cX"] \dar["\cong" left] & \rH^{n+k}(\cX; G) \dar["\cong" right] \\
	\rH^{n+1}(\sus \cX; \pi) \rar["\theta^{n+1}_\cX"] & \rH^{n+1+k}(\sus \cX; G)
\end{tikzcd}
\]
commute for each \(\cX\) and $n$.
We will denote by $\cO^{\mathrm{st}}_k(\pi, G)$ the set of stable cohomology operation of degree $k$.

%An example of a cohomology operation that is not stable is the cup-product square
%\[
%\Big\{{\begin{tikzcd}[column sep=small,row sep=-6pt]
%	\rH^n(\cX) \rar & \rH^{2n}(\cX) \\
%	x \rar[maps to] & x^2
%\end{tikzcd}}\Big\}
%\]
%ranging over all $n$.\footnote{To see this notice that suspending this map gives a map $\rH^{n+1}(\Sigma \cX) \to \rH^{2n+1}(\Sigma \cX)$, which has the wrong degree.
	%In fact, on $\rH^*(\Sigma \cX)$, all cup products vanish.}

Using the representability of cohomology, $\cO^{\mathrm{st}}_k(\pi, G)$ can be described as the colimit of the groups
\[
\rH^{n+k}(K(\pi, n); G) \cong [K(\pi, n), K(G, n+k)]
\]
along certain maps
\[
[K(\pi, n), K(\pi, n+k)] \to [K(\pi, n+1), K(G, n+1+k)]
\]
defined by the functoriality of $\sus$ and a natural homotopy equivalence
\begin{equation}\label{eq:suspension of K(pi,n)}
	\sus K(\pi, n) \cong K(\pi, n+1).
\end{equation}
%between the suspension of the $n^\th$ Eilenberg--MacLane space and the $(n+1)^\th$.\footnote{\anibal{include spectra?}}
In other words, we have
\[
\cO^\st_k(\pi, G) \cong \colim_n \rH^{n+k}(K(\pi, n); G).
\]

\subsubsection{} We will be interested in combining all stable operations over a fixed coefficient field $\k$ to form, via compositions, the (graded) algebra:
\[
\cO^\st(\k) = \bigsqcup_k \cO^{\mathrm{st}}_k(\k, \k).
\]

For any prime $p$ let \(\Fp\) be the field with \(p\) elements.
The algebra $\cO^\st(\Fp)$ is denoted $\cA_p$ and referred to as the mod~$p$ \defn{Steenrod Algebra}.
%Its importance in stable homotopy theory is hard to overstate.
%When $\k$ is $\Z/p$ for some prime $p$, the algebra of stable cohomology operations is known as the mod $p$ \defn{Steenrod Algebra} and is of central importance in homotopy theory.
%In the computational part of this paper, we will focus on the mod 2 case, referring to $\cO^{\mathrm{st}}(\Z/2)$ simply as the Steenrod Algebra.

%In terms of the representing object, this is an isomorphism
%\[
%[\cX, K(\pi, n)] \cong [\Sigma \cX, K(\pi, n+1)].
%\]
%The right-hand side is the same as \( [\cX, \Omega K(\pi, n+1)] \), so the suspension isomorphism is induced by the homotopy equivalence \( K(\pi, n) \simeq \Omega K(\pi, n+1) \).
%
%Again using representability, the set of stable cohomology operations will be the colimit of the groups
%\[
%[K(\pi, n), K(\pi, n+k)]
%\]
%along the maps
%\[
%[K(\pi, n), K(\pi, n+k)] \cong [\Omega K(\pi, n+1), \Omega K(\pi, n+k+1)] \cong [\Sigma \Omega K(\pi, n+1), K(\pi, n+k+1)] \to [K(\pi, n+1), K(\pi, n+k+1)].
%\]
%Equivalently, I want the colimit
%\[
%\varinjlim_n \rH^{n+k}(K(\pi, n))
%\]
%along the maps
%\[
%K(\pi, n+1) \to \Sigma \Omega K(\pi, n+1) \simeq \Sigma K(\pi, n).
%\]
%This colimit is the group of degree-\( n \) stable cohomology operations.
%Taking this for all \( n \), we get a graded ring (I can compose two stable cohomology operations to get another one). This ring is the Steenrod Algebra.

\subsubsection{}

The Steenrod algebra $\cA_2$ is generated by the \defn{Steenrod squares} $\set{\Sq^i}_{i \in \N}$, where $\Sq^i \colon \rH^n(\cX; \Ftwo) \to \rH^{n+i}(\cX; \Ftwo)$ for any $n\geq 0$.
These are degree $i$ stable operations which are axiomatically characterized by the following properties holding for all cohomology classes $\alpha,\beta \in \rH^*(\cX; \Ftwo)$ in any space $\cX$:
\begin{enumerate}
	\item \(\Sq^0(\alpha) = \alpha,\)
	\item \(\Sq^i(\alpha) = 0, \quad \text{if } i > \deg\alpha.\)
	\item \(\Sq^i(\alpha) = \alpha^2, \quad \text{if } i = \deg\alpha.\)
	\item \(\Sq^i(\alpha \beta) = \textstyle\sum_{j=0}^{i} \Sq^j(\alpha) \Sq^{i-j}(\beta).\)
\end{enumerate}
We mention that $\Sq^1$ is the Bockstein homomorphism of the exact sequence
\[
0 \to \Z/2 \to \Z/4 \to \Z/2 \to 0.
\]

The \defn{total Steenrod square} is the (inhomogeneous) cohomology operation
\[
\Sq = \Sq^0 + \Sq^1 + \Sq^2 + \dotsb,
\]
acting on \(\bigoplus_{m \in \N} \rH^m(\cX; \Ftwo)\), where each \(\Sq^i\) represents an individual Steenrod square operation.

\subsubsection{}\label{sss:cohomology_rpn}

Consider the system of equatorial inclusions
\[
\bS^0 \subset \bS^1 \subset \bS^2 \subset \dotsb
\]
together with the (compatible) antipodal action on spheres.
A model for \(K(\Z/2, 1)\) is \(\rp^\infty\), the union of the resulting orbits
\[
\rp^0 \subset \rp^1 \subset \rp^2 \subset \dotsb.
\]
%These spaces are of central in the study of the mod~2 Steenrod algebra.

The cohomology algebra of $\rp^n$ with mod 2 coefficients is the polynomial algebra generated by a single element $\sigma$ in degree 1.
That is to say,
\[
\rH^\ast(\rp^\infty; \Ftwo) \cong \Ftwo[\sigma].
\]
Additionally, for any $n \in \N$,
\[
\rH^\ast(\rp^n; \Ftwo) \cong \frac{\Ftwo[\sigma]}{(\sigma^{n+1} = 1)}.
\]

The action of the Steenrod algebra $\cA_2$ on $\rH^*(\rp^n, \Ftwo)$, for $n$ possibly equal to $\infty$, is either the 0 map or is given by
\[
\Sq^k(\sigma^\ell) = \binom{\ell}{k}\sigma^{\ell+k}
\]
for $0 \leq k \leq \frac{n-1}{2}$ and $k \leq \ell$.
Here the binomial coefficient is taken over \(\Ftwo\).

\subsubsection{} For an odd prime \(p\), the Steenrod algebra $\cA_p$ is generated by the \defn{Bockstein} homomorphism of the exact sequence
\[
0 \to \Z/p \to \Z/p^2 \to \Z/p \to 0
\]
and the \defn{Steenrod reduced powers} \(\set{\rP^i}_{i \in \N}\), where $\rP^i \colon \rH^n(\cX; \Fp) \to \rH^{n+2i(p-1)}(\cX; \Fp)$, for any $n\geq 0$.
These are degree \(2i(p-1)\) stable operations which are axiomatically characterized by the following properties holding for all cohomology classes \(\alpha, \beta \in \rH^*(\cX; \Fp)\) in any space \(\cX\):

\begin{enumerate}
	\item \(\rP^0(\alpha) = \alpha,\)
	\item \(\rP^i(\alpha) = 0, \quad \text{if } 2i > \deg\alpha,\)
	\item \(\rP^i(\alpha) = \alpha^p, \quad \text{if } 2i = \deg\alpha,\)
	\item \(\rP^i(\alpha \beta) = \textstyle\sum_{j=0}^{i} \rP^j(\alpha) \rP^{i-j}(\beta).\)
\end{enumerate}

The \defn{total Steenrod reduced power} is the (inhomogeneous) cohomology operation
\[
\rP = \rP^0 + \rP^1 + \rP^2 + \dotsb,
\]
acting on \(\bigoplus_{m \in \N} \rH^m(\cX; \Fp)\), where each \(\rP^i\) represents an individual Steenrod reduced power.

\subsubsection{Lens Spaces}

