% !TEX root = ../vr_st.tex

\subsection{General quotients}

\subsubsection{}

Let $G$ be a group acting on a metric space $\cX$, we denote its orbit space with the quotient topology by $\cX_G$.
For $x \in \cX$, its orbits will be denoted as $[x]$.
We say the action of $G$ is \defn{proper} if, for every $x \in \cX$, there is some $r>0$ such that $\{g \mid g\cdot B(x,r) \cap B(x,r) = \emptyset\}$ is finite.
We say $G$ \defn{acts by isometries} on $\cX$, if the map $g \colon \cX \to \cX$ is an isometry for every $g \in G$.

Let $G$ be a group acting properly and by isometries on a metric space $\cX$.
Then the \defn{quotient metric}, defined by
\[
d_{\cX_G}\big([x], [x']\big) = \inf_{g \in G} d_\cX(x, g \cdot x'),
\]
is well-defined on $\cX_G$.

\subsubsection{}\label{ss:h}

A \(G\)-action on $\cX$ induces a natural \(G\)-action on the Vietoris--Rips complex $\VR(\cX)$.
Explicitly, for any \(r > 0, g\in G\) and $\sum \lambda_i x_i \in \VR_r(\cX)$,
\[
g \cdot \sum \lambda_i x_i = \sum \lambda_i (g\cdot x_i).
\]
With respect to this action, there is an induced map on orbit spaces given by
\begin{align*}
	\tilde h_r \colon \VR_r(\cX_G) &\to \VR_r(\cX)_G \\
	\textstyle\sum\lambda_i [x_i] &\mapsto \textstyle \big[\sum\lambda_i x_i\big].
\end{align*}
If the action is proper and by isometries, as well as satisfies the \textit{strong \(r\)-diameter} condition, a concept we define below, then for any \(s \leq r\), the map $\tilde{h}_s$ is an isomorphism of simplicial complexes.
This result was originally stated in \cite[Prop.~3.5]{adams2022metric} under a assumption weaker than strong \(r\)-diameter.
We learned about a counterexample to the original statement, along with the stronger condition that corrects it, through direct communication with Benjamin Barham regarding his forthcoming work on the subject.

\subsubsection{}\label{sss:strong_r_action}

The action of a group $G$ on $\cX$ is said to be a \defn{strong $r$-diameter action}, where $r > 0$, if for any non-negative integer $k$, the condition $\diam_{\cX_G}\set[\big]{[x_0],\dots,[x_k]} < r$ implies the existence of a unique choice of $g_i$ for each $i \in \set{1,\dots,k}$ such that $\diam_{\cX}\set{x_0, g_1x_1, \dots, g_kx_k} < r$.
It is not hard to see that a strong $r$-diameter action is free.

Recall from \cref{sss:cohomology_rpn} and \cref{sss:cohomology_lens} the canonical $\rC_2$- and $\rC_q$-actions on spheres, for $q \in \N$, that define real projective spaces and Lens spaces, respectively.

\medskip\lemma
The canonical $\rC_2$ and $\rC_q$ actions on round unit spheres are strong $r$-diameter for every $0 < r \leq \tfrac{2\pi}{q(q+1)}$.

\begin{proof}
    For $\rC_2$, this is established in \cite[Cor.~4.3]{adams2022metric}. We extend the proof to the case of $\rC_q$ for \(q \geq 2\).

    Let $\bS$
%    $G = \rC_q$,
    be an odd-dimensional round unit sphere
%    with a canonical $G$-action,
	and $\omega = e^{2\pi i/q}$.
%	the
%	primitive $q$-th root of unity.
	Assume $\diam_{\rL_q}\{[x_0], \dots, [x_k]\} < r$.
	Then, for each $i \in \{1, \dots, k\}$, there exists $g_i \in \rC_q$ such that $d_{\bS}(x_0, g_i x_i) < r$. Let $x_i^* = g_i x_i$. We claim that $d_{\bS}(x_i^*, x_j^*) < r$ for all $1 \leq i, j \leq k$.
%    Assume not.
    If this is not the case, then there exists $0 < a < q-1$ such that $d_{\bS}(x_i^*, \omega^a x_j^*) < r$.
    This leads to the following
%    a
	contradiction:
    \begin{align*}
    \frac{2\pi}{q} \,=\, d_{\bS}(x_0, \omega x_0) \leq\, & \, d_{\bS}(x_0, x_i^*) \, + \, d_{\bS}(x_i^*, \omega^a x_j^*) \, + \, \\
    & \sum_{b=1}^a d_{\bS}(\omega^b x_j^*, \omega^{b-1} x_j^*) \, + \, d_{\bS}(\omega x_j^*, \omega x_0).
    \end{align*}
    The right-hand side is \emph{strictly} bounded above by $r \cdot (a + 2) \leq r \cdot (q + 1) \leq \frac{2\pi}{q}$, a contradiction. Thus, $d_{\bS}(x_i^*, x_j^*) < r$ holds, completing the proof.\anibal{I think \(b=1\) was over-counted.}

    We still need to verify the uniqueness of such $\{g_i\}$.
    This is because %straightforward, because \(r \leq \tfrac{2\pi}{q(q+1)} < \frac{1}{2} d_{\bS}(\omega x_i^*, x_i^*)\).
    %Therefore,
    $x_i^*$ and $gx_i^*$, for $g$ a non-identity element, cannot simultaneously satisfy $d_{\bS}(x_0, x_i^*) < r$ and $d_{\bS}(x_0, gx_i^*) < r$.
    If they were, then we would have
    \[\tfrac{2\pi}{q} \leq d_{\bS}(x_i^*, gx_i^*) \leq d_{\bS}(x_i^*, x_0) + d_{\bS}(x_0, gx_i^*) < 2r \leq \tfrac{2\pi}{q} \cdot \tfrac{2}{q+1},
    \]
    contracting
%    with
	the assumption $q \geq 2$.
\end{proof}

A well-known action that is not strong \(r\)-diameter for any \(r>0\) is the circle action on spheres defining complex projective spaces.

\subsection{Quotients of spheres}

\subsubsection{}\label{ss:VR-compatible-Sn}

%We will be interested in round spherical quotients.
%These are manifolds obtained from the quotient of a round sphere by a compact group acting freely, properly, and by isometries.
%For us the groups will be always finite.

A proper \(G\)-action by isometries on \(\bS^n\) is \defn{\(\VR\)-compatible} if there exists \(r_0 > 0\) such that the action is strong \(r\)-diameter for all \(0 < r < r_0\), for each \(r \in (0, \zeta_n]\) \anibal{I am surprise it is \(\zeta_n\) instead of \(\zeta_n/2\) which is the critical radii} it commutes with the canonical projection of \(\VR_r(\bS^n)\) (see \cref{ss:VRSn projection}), and the induced map
\[
\tilde f_r^n \colon \VR_r(\bS^n)_G \to \bS^n_G
\]
is then a weak equivalence.\anibal{I think this 3er one is an additional property. I changed it from phrasing claiming that the first 2 conditions imply it.}
For example, based on the results in \cref{ss:VRSn projection}, the trivial action is \(\VR\)-compatible.

\lz{Strong $r$-diameter implies free (\cite{adams2022metric} mentioned so with the weaker definition). Does it imply discrete or finite? proper?}

\subsubsection{}\label{subsub:VR-compatible-system}

An \defn{equatorial system} is a diagram
\[
\bS^{n_1} \to \bS^{n_2} \to \bS^{n_3} \to \dotsb
\]
of round spheres where each map is an isometric embedding.
For example, the real and complex equatorial systems
\[
\bS^1 \subset \bS^2 \subset \bS^3 \subset \dotsb
\quad\text{and}\quad
\bS^1 \subset \bS^3 \subset \bS^5 \subset \dotsb,
\]
used for the definition of real projective (\cref{sss:cohomology_rpn}) and Lens spaces (\cref{sss:cohomology_lens}), are defined by the canonical inclusions
\[
\R^2 \subset \R^3 \subset \R^4 \subset \dotsb
\quad\text{and}\quad
\bC \subset \bC^2 \subset \bC^3 \subset \dotsb.
\]

\subsubsection{}\label{ss:system VR compatible}

A \(G\)-action on an equatorial system consists of a \(G\)-action on each sphere commuting with the isometric embeddings.
It is said to be free, proper, or by isometries if the \(G\)-action on every sphere is.

The group \(\rC_2 = \set{1,-1} \subset \R\) acts on the real equatorial system, and, for \(q \in \N\), the multiplicative subgroup \(\rC_q\) of \(q\)-roots of unity acts on the complex one.
Both of these actions are free, proper, and by isometries.

We say that a proper \(G\)-action by isometries on an equatorial system is \defn{\(\VR\)-compatible} if the \(G\)-action on each sphere is \(\VR\)-compatible and the following diagram commutes for every \(i \in \N\) and $0 < r < \zeta_{n_{i+1}}$:\anibal{\(\zeta_n\) decreases with \(n\), right?}
\begin{equation}\label{eq:VR_quotient}
    \begin{tikzcd}
	\VR_r(\bS^{n_i})_G
	\ar[d]
	\ar[r, "\tilde f_{\,r}^{\,n_i}" above]
	&
	\bS^{n_i}_G
	\ar[d]
	\\
	\VR_r(\bS^{n_{i+1}})_G
	\ar[r, "\tilde f_{\,r}^{\,n_{i+1}}" above]
	&
	\bS^{n_{i+1}}_G.
\end{tikzcd}
\end{equation}

\lemma The above \(\rC_2\)-action on the real equatorial system and the \(\rC_q\)-action on the complex one are \(\VR\)-compatible for every \(q \in \N\).

\begin{proof}
    Let $n \in \N$ and take any $0 < r \leq \zeta_{n+1} (< \zeta_{n})$.
    The commutativity of Diagram~\ref{eq:VR_quotient} with $n_i = n$ and $n_{i+1} = n+1$ follows directly from the commutativity of the analogous diagram without group actions.

    Next, we verify that the group action on each sphere $\bS^n$ is \(\VR\)-compatible (see \cref{ss:VR-compatible-Sn}).
    By \cref{sss:strong_r_action}, these actions are strong \(r\)-diameter actions for small enough $r>0$.
    We need to verify that the group action commutes with the $\VR_r(\bS^{n})$-projection $f_r^{n}\colon \VR_r(\bS^{n}) \to \bS^{n}$ and that $f_r^{n}$ induces weak equivalences on the orbit spaces for all $0 < r \leq \zeta_{n}$.

    For the $\rC_2$-action on the real equatorial system, verifying that the $\rC_2$-action commutes with $f_r^{n}$ is straightforward.
    By \cref{ss:VRSn projection}, $f_r^{n}$ is a homotopy equivalence.
    Because the $\rC_2$-action is proper and free, the $\rC_2$-equivariant homotopy equivalence $f_r^n$ induces homotopy equivalence (and thus weak equivalence) on the orbit spaces.

    The case of $\rC_q$-action on the complex equatorial system follows similarly.
\end{proof}

\subsubsection{}\label{ss:fundamental_lemma}

\medskip\lemma Consider a $\VR$-compatible \(G\)-action on an equatorial system
\[
\bS^{n_1} \to \bS^{n_2} \to \bS^{n_3} \to \dotsb .
\]
If $\fillrad(\bS^{n_i}_G)$ is non-decreasing as a function of \(i\) then, for any \(i \leq j\),
\[
\firstdeath{n_i}{\bS^{n_j}_G} \leq \fillrad(\bS^{n_j}_G).
\]

\begin{proof}
    To simplify notation, for any $i \leq j$, let
    \[
        \cX_j = \bS^{n_j}_G, \,\delta_j = \fillrad(\cX_j) \text{ and }\beta_i^j = \firstdeath{n_i}{\cX_j}.
    \]
	We will use an induction argument on $j$.
	When $j = 1$, because $\cX_1$ is connected and $n_1$-dimensional, we apply results in \cref{ss:beta v.s. fillrad} to deduce that $\beta_1^1 = \delta_1$.

	Assume the statement holds for $\cX_{j-1}$, that is, $\beta_i^{j-1} \leq \delta_{j-1}$ for any $i \leq j-1$.
	Since $\delta_{j-1} \leq \delta_j$, we have the following commutative diagram of topological spaces for any $r,\epsilon>0$ small:
    \begin{equation}\label{d:fundamental_bars_diagram}
        \begin{tikzcd}
            \cX_{j-1}
            \ar[d, hook,"{\iota}" left]
            &
            \VR_r(\cX_{j-1})
            \ar[d, hook,"\iota_r" left]
            \ar[l, "\rho^{j-1}" above]
            \ar[r, hook, "v^{j-1}"]
            &
            \VR_{\delta_{j-1}+\epsilon}(\cX_{j-1})
            \ar[d, hook]
            \\
            \cX_j
            &
            \VR_r(\cX_j)
            \ar[l, "\rho^j" below]
            \ar[r, hook, "v^j" below]
            &
            \VR_{\delta_j+\epsilon}(\cX_j).
        \end{tikzcd}
    \end{equation}
    Here, the vertical maps are all induced by the equatorial inclusion of the orbit spaces $\iota \colon \cX_{j-1} \hookrightarrow \cX_j$.
    The horizontal inclusion $v^{j-1}$ (resp. $v^j$) in the right-hand-side square is the inclusion map in the corresponding Vietoris--Rips filtration.
    The horizontal map $\rho^{j-1}$ in the left-hand-side square is the composition of the following maps introduced in \cref{ss:h} and \cref{ss:VRSn projection}, respectively:
    \[\VR_r(\cX_{j-1}) \xrightarrow{\tilde{h}_r}\VR_r(\bS^{n_{j-1}})_G \xrightarrow{\tilde{f}_r^{n_{j-1}}} \cX_{j-1}.\]
    Since the \( G \)-action on \( \bS^{n_{j-1}} \) is VR-compatible, it is a strong \( r \)-diameter action for small enough $r>0$, making \( \tilde{h}_r \) an isomorphism by \cref{ss:h}.
    This VR-compatibility further implies that \( \tilde{f}_r^{n_{j-1}} \) is a weak equivalence, and therefore \( \rho^{j-1} \) is as well.
    The horizontal map \( \rho^j \), defined similarly, is likewise a weak equivalence.

    For any $i \leq j-1$, applying the degree $n_i$ reduced homology functor to diagram (\ref{d:fundamental_bars_diagram}) and using the fact that the $\rC_2$-action on the system is $\VR$-compatible (cf. \cref{ss:system VR compatible}), we obtain a commutative diagram of vector spaces:
	for $r,\epsilon>0$ small,
	\begin{equation}\label{eq:diagram of H}
	\begin{tikzcd}[column sep = 4.5em]
		\rH_{n_i}(\cX_{j-1})
		\ar[d, "\cong" left]
		&
		\rH_{n_i}\big(\VR_r(\cX_{j-1})\big)
		\ar[d, "\rH_{n_i}(\iota_r)" left, "\cong" right, myred]
		\ar[l, "\cong" above]
        %\ar[l, "(1)" below]
		\ar[r, "\rH_{n_i}(v^{{j-1}})", myred]
		&
		\rH_{n_i}\big(\VR_{\delta_{j-1}+\epsilon}(\cX_{j-1})\big)
		\ar[d]
		\\
		\rH_{n_i}(\cX_j)
		&
		\rH_{n_i}\big(\VR_r(\cX_j)\big)
		\ar[l, "\cong"]
        %\ar[l, "(2)" above]
		\ar[r, "\rH_{n_i}(v^j)" below, myred]
		&
		\rH_{n_i}\big(\VR_{\delta_n+\epsilon}(\cX_j)\big).
	\end{tikzcd}
	\end{equation}

	Let $\sigma_i$ be a representative cycle for the bar $(0,\, 2\beta_{i}^{j-1})$ in $\VR(\cX_{j-1})$.
	Commutativity of the left-hand-side square of diagram (\ref{eq:diagram of H}) implies that $\rH_{n_i}(\iota_r)$ is an isomorphism.
    As $r$ is arbitrarily small, we obtain that $\rH_{n_i}(\iota_r)(\sigma_i)$ creates a bar born at $0$.
	Moreover, this bar dies after $\delta_j$ as explained below.
    It follows from the induction hypothesis $\beta_i^{j-1} \leq \delta_{j-1}$ that $\sigma_i$ dies after $\delta_{j-1}$, implying that $\rH_{n_i}(v^{{j-1}})(\sigma_i) = 0$.
	Using the right-hand-side square's commutativity, we have $(\rH_{n_i}(v^j) \circ \rH_{n_i}(\iota_r))(\sigma_i)=0$, which means $\rH_{n_i}(\iota_r)(\sigma_i)$ dies after $\delta_j+\epsilon$.
    Since this holds for any \(\epsilon > 0\), we conclude that \(\beta_i^j \leq \delta_j\).

	For the case when $i = j$, apply results in \cref{ss:beta v.s. fillrad} again to get that $\beta_j^j = \delta_j$.
	This completes the proof.
\end{proof}