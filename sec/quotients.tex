% !TEX root = ../vr_st.tex

\subsection{General quotients}

\subsubsection{}

%An action of a group $G$ on a set $X$ is a function $G\times X\to X$ such that $g\cdot (h\cdot x)=(gh)\cdot x$, and $e\cdot x = x$ for all $x\in X$ and $g,h\in G$, where $e$ is the identity element of $G$.

Let $G$ be a group acting on a metric space $\cX$, we denote its orbit space with the quotient topology by $\cX_G$.
Orbits will be denoted as $[x]$ for $x \in \cX$.
We say the action of $G$ is \defn{proper} if, for every $x \in \cX$, there is some $r>0$ such that $\{g \mid g\cdot B(x,r) \cap B(x,r) = \emptyset\}$ is finite.
We say $G$ \defn{acts by isometries} on $\cX$, if the map $g \colon \cX \to \cX$ is an isometry for every $g \in G$.

Let $G$ be a group acting properly and by isometries on a metric space $\cX$.
Then the \defn{quotient metric}, defined by
\[
d_{\cX_G}\big([x], [x']\big) = \inf_{g \in G} d_\cX(x, g \cdot x'),
\]
is well-defined on $\cX_G$.

The action of a group $G$ on $\cX$ is said to be a \defn{strong $r$-diameter action}, where $r > 0$, if for any non-negative integer $k$, the condition $\diam_{\cX_G}\set[\big]{[x_0],\dots,[x_k]} < r$ implies the existence of a unique choice of $g_i$ for each $i \in \set{1,\dots,k}$ such that $\diam_{\cX}\set{x_0, g_1x_1, \dots, g_kx_k} < r$.
\lz{I think the citation of Liam should go right after the definition.}

% \medskip\remark
% If the action of $G$ on $\cX$ is a strong $r$-diameter action for some $r > 0$, then $G$ acts freely on $\cX$.

\subsubsection{}\label{subsub:h}

A \(G\)-action on $\cX$ induces a natural \(G\)-action on the Vietoris--Rips complex $\VR(\cX)$.
Explicitly, for any \(r > 0, g\in G\) and $\sum \lambda_i x_i \in \VR_r(\cX)$,
\[
g \cdot \sum \lambda_i x_i = \sum \lambda_i (g\cdot x_i).
\]
With respect to this action, there is an induced map on orbit spaces given by
\begin{align*}
	\tilde h_r \colon \VR_r(\cX_G) &\to \VR_r(\cX)_G \\
	\textstyle\sum\lambda_i [x_i] &\mapsto \textstyle \big[\sum\lambda_i x_i\big].
\end{align*}
If the action is proper and by isometries as well as strong \(r\)-diameter, then for any \(s \leq r\), the map $\tilde{h}_s$ is an isomorphism of simplicial complexes.
This result was initially stated in \cite[Prop.~3.5]{adams2022metric} under a weaker assumption than the strong \(r\)-diameter condition.
A counterexample to the original statement and the above corrected one were provided by Benjamin Barham through direct communication of his upcoming work.


\subsubsection{} \label{sss:strong_r_action_examples}
\lz{this section is newly added}

Recall from \cref{sss:cohomology_rpn} and \cref{sss:cohomology_lens} the $\rC_2$ and $\rC_q$ actions on spheres, for $ q\in \N$, that define real projective spaces and lens spaces, respectively. 

\medskip\lemma
The above $\rC_2$ and $\rC_q$ actions on spheres are strong $r$-diameter actions for every $r\in (0, \zeta_n]$.

\begin{proof}
    For $\rC_2$, this is proved in \cite[Cor.~4.3]{adams2022metric}.

    For $\rC_q$, \lz{to fill in}
\end{proof}

Complex projective spaces are quotients of spheres by the circle group.
We note that, unlike the above finite group actions that form the real projective spaces and lens spaces, these circle group actions on spheres is not strong $r$-diameter actions for any $r\in (0, \zeta_n]$. 
\lz{Do we need to write the proof for it?}

\subsection{Quotients of spheres}

\subsubsection{}\label{ss:VR-compatible-Sn}

We will be interested in round spherical quotients.
These are manifolds obtained from the quotient of a round sphere by a compact group acting freely, properly, and by isometries.
For us the groups will be always finite.\anibal{Remain myself why complex projective spaces did not work for us.}

A \(G\)-action which acts properly by isometries on \(\bS^n\) \lz{I am not sure if it is a good idea to remove `proper', because `acts properly by isometries' is a commonly used terminology to guarantee the quotient metric is well-defined.} is said to be \defn{\(\VR\)-compatible} if, for each $r \in (0, \zeta_n]$, it is a strong $r$-diameter action and it commutes with the canonical projection of \(\VR_r(\bS^n)\) (see \cref{ss:VRSn projection}) and the induced map
\[
\tilde f_r^n \colon \VR_r(\bS^n)_G \to \bS^n_G,
\]
is a weak equivalence.
For example, based on the results in \cref{ss:VRSn projection}, the trivial action is \(\VR\)-compatible.

\lz{Strong $r$-diameter implies free (\cite{adams2022metric} mentioned so with the weaker definition). Does it imply discrete or finite? proper?}

\subsubsection{}\label{subsub:VR-compatible-system}

An \defn{equatorial system} is a diagram
\[
\bS^{n_1}(\rho) \to \bS^{n_2}(\rho) \to \bS^{n_3}(\rho) \to \dotsb
\]
of round spheres where each map is an isometric embedding.
For example, the real and complex equatorial systems
\[
\bS^1 \subset \bS^2 \subset \bS^3 \subset \dotsb
\quad\text{and}\quad
\bS^1 \subset \bS^3 \subset \bS^5 \subset \dotsb,
\]
used for the definition of real projective (\cref{sss:cohomology_rpn}) and Lens spaces (\cref{sss:cohomology_lens}), are defined by the canonical inclusions
\[
\R^2 \subset \R^3 \subset \R^4 \subset \dotsb
\quad\text{and}\quad
\bC \subset \bC^2 \subset \bC^3 \subset \dotsb.
\]

\subsubsection{}\label{ss:system VR compatible}

A \(G\)-action on an equatorial system consists of a \(G\)-action on each sphere commuting with the isometric embeddings.
It is said to be free, proper, or by isometries if the \(G\)-action on every sphere is.

The group \(\rC_2 = \set{1,-1} \subset \R\) acts on the real equatorial system, and, for \(q \in \N\), the multiplicative subgroup \(\rC_q\) of \(q\)-roots of unity acts on the complex one.
Both of these actions are free, proper, and by isometries.

We say that a free and proper \(G\)-action by isometries on an equatorial system is \defn{\(\VR\)-compatible} if the \(G\)-action on each sphere is \(\VR\)-compatible and the following diagram commutes for every \(i \in \N\) and $0 < r < \zeta_{n_{i+1}}$:
\begin{equation}\label{eq:VR_quotient}
    \begin{tikzcd}
	\VR_r(\bS^{n_i})_G
	\ar[d]
	\ar[r, "\tilde f_{\,r}^{\,n_i}" above]
	&
	\bS^{n_i}_G
	\ar[d]
	\\
	\VR_r(\bS^{n_{i+1}})_G
	\ar[r, "\tilde f_{\,r}^{\,n_{i+1}}" above]
	&
	\bS^{n_{i+1}}_G.
\end{tikzcd}
\end{equation}

\lemma The above \(\rC_2\)-action on the real equatorial system and the \(\rC_q\)-action on the complex one are \(\VR\)-compatible for every \(q \in \N\).

\begin{proof}
    Let $n \in \N$ and take any $0 < r \leq \zeta_{n+1} (< \zeta_{n})$.
    The commutativity of diagram \ref{eq:VR_quotient} with $n_i = n$ and $n_{i+1} = n+1$ follows directly from the commutativity of the analogous diagram without group actions.

    Next, we verify that the group action on each sphere $\bS^n$ is \(\VR\)-compatible (see \cref{subsub:VR-compatible-Sn}).
    By \cref{sss:strong_r_action_examples}, these actions are strong \(r\)-diameter actions. \lz{This paragraph is newly edited.}
    Therefore, it remains to verify that the group action commutes with the $\VR_r(\bS^{n})$-projection $f_r^{n}\colon \VR_r(\bS^{n}) \to \bS^{n}$ and that $f_r^{n}$ induces weak equivalences on the orbit spaces.

    For the $\rC_2$-action on the real equatorial system, verifying that the $\rC_2$-action commutes with $f_r^{n}$ is straightforward.
    By \cref{ss:VRSn projection}, $f_r^{n}$ is a homotopy equivalence.
    Because the $\rC_2$-action is proper and free, the $\rC_2$-equivariant homotopy equivalence $f_r^n$ induces homotopy equivalence (and thus weak equivalence) on the orbit spaces.

    The case of $\rC_q$-action on the complex equatorial system follows similarly.
\end{proof}

\subsubsection{}\label{ss:fundamental_lemma}

\medskip\lemma Consider a $\VR$-compatible \(G\)-action on an equatorial system
\[
\bS^{n_1} \to \bS^{n_2} \to \bS^{n_3} \to \dotsb.
\]
If $\fillrad(\bS^{n_i}_G)$ is non-decreasing as a function of \(i\) then, for any \(i \leq j\),
\[
\firstdeath{n_i}{\bS^{n_j}_G} \leq \fillrad(\bS^{n_j}_G).
\]

\begin{proof}
    To simplify notation, for any $i \leq j$, let
    \[
        \cX_j = \bS^{n_j}_G, \,\delta_j = \fillrad(\cX_j) \text{ and }\beta_i^j = \firstdeath{n_i}{\cX_j}.
    \]
	We will use an induction argument on $j$.
	When $j = 1$, because $\cX_1$ is connected and $n_1$-dimensional, we apply results in \cref{ss:beta v.s. fillrad} to deduce that $\beta_1^1 = \delta_1$.

	Assume the statement holds for $\cX_{j-1}$, that is, $\beta_i^{j-1} \leq \delta_{j-1}$ for any $i \leq j-1$.
	Since $\delta_{j-1} \leq \delta_j$, we have the following commutative diagram of topological spaces for any $r,\epsilon>0$ small:
    \begin{equation}\label{d:fundamental_bars_diagram}
        \begin{tikzcd}
            \cX_{j-1}
            \ar[d, hook,"{\iota}" left]
            &
            \VR_r(\cX_{j-1})
            \ar[d, hook,"\iota_r" left]
            \ar[l, "\rho^{j-1}" above]
            \ar[r, hook, "v^{j-1}"]
            &
            \VR_{\delta_{j-1}+\epsilon}(\cX_{j-1})
            \ar[d, hook]
            \\
            \cX_j
            &
            \VR_r(\cX_j)
            \ar[l, "\rho^j" below]
            \ar[r, hook, "v^j" below]
            &
            \VR_{\delta_j+\epsilon}(\cX_j).
        \end{tikzcd}
    \end{equation}
    Here, the vertical maps are all induced by the equatorial inclusion of the orbit spaces $\iota \colon \cX_{j-1} \hookrightarrow \cX_j$.
    The horizontal inclusion $v^{j-1}$ (resp. $v^j$) in the right-hand-side square is the inclusion map in the corresponding Vietoris--Rips filtration.
    The horizontal map $\rho^{j-1}$ in the left-hand-side square is the composition of the following maps introduced in \cref{ss:h} and \cref{ss:VRSn projection}, respectively:
    \[\VR_r(\cX_{j-1}) \xrightarrow{\tilde{h}_r}\VR_r(\bS^{n_{j-1}})_G \xrightarrow{\tilde{f}_r^{n_{j-1}}} \cX_{j-1}.\]
    \lz{this paragraph is newly edited:} Since the \( G \)-action on \( \bS^{n_{j-1}} \) is VR-compatible, it is also a strong \( r \)-diameter action, making \( \tilde{h}_r \) an isomorphism by \cref{subsub:h}. 
    This VR-compatibility further implies that \( \tilde{f}_r^{n_{j-1}} \) is a weak equivalence, and therefore \( \rho^{j-1} \) is as well. 
    The horizontal map \( \rho^j \), defined similarly, is likewise a weak equivalence.

    For any $i \leq j-1$, applying the degree $n_i$ reduced homology functor to diagram (\ref{d:fundamental_bars_diagram}) and using the fact that the $\rC_2$-action on the system is $\VR$-compatible (cf. \cref{ss:system VR compatible}), we obtain a commutative diagram of vector spaces:
	for $r,\epsilon>0$ small,
	\begin{equation}\label{eq:diagram of H}
	\begin{tikzcd}[column sep = 4.5em]
		\rH_{n_i}(\cX_{j-1})
		\ar[d, "\cong" left]
		&
		\rH_{n_i}\big(\VR_r(\cX_{j-1})\big)
		\ar[d, "\rH_{n_i}(\iota_r)" left, "\cong" right, myred]
		\ar[l, "\cong" above]
        %\ar[l, "(1)" below]
		\ar[r, "\rH_{n_i}(v^{{j-1}})", myred]
		&
		\rH_{n_i}\big(\VR_{\delta_{j-1}+\epsilon}(\cX_{j-1})\big)
		\ar[d]
		\\
		\rH_{n_i}(\cX_j)
		&
		\rH_{n_i}\big(\VR_r(\cX_j)\big)
		\ar[l, "\cong"]
        %\ar[l, "(2)" above]
		\ar[r, "\rH_{n_i}(v^j)" below, myred]
		&
		\rH_{n_i}\big(\VR_{\delta_n+\epsilon}(\cX_j)\big).
	\end{tikzcd}
	\end{equation}

	Let $\sigma_i$ be a representative cycle for the bar $(0,\, 2\beta_{i}^{j-1})$ in $\VR(\cX_{j-1})$.
	Commutativity of the left-hand-side square of diagram (\ref{eq:diagram of H}) implies that $\rH_{n_i}(\iota_r)$ is an isomorphism.
    As $r$ is arbitrarily small, we obtain that $\rH_{n_i}(\iota_r)(\sigma_i)$ creates a bar born at $0$.
	Moreover, this bar dies after $\delta_j$ as explained below.
    It follows from the induction hypothesis $\beta_i^{j-1} \leq \delta_{j-1}$ that $\sigma_i$ dies after $\delta_{j-1}$, implying that $\rH_{n_i}(v^{{j-1}})(\sigma_i) = 0$.
	Using the right-hand-side square's commutativity, we have $(\rH_{n_i}(v^j) \circ \rH_{n_i}(\iota_r))(\sigma_i)=0$, which means $\rH_{n_i}(\iota_r)(\sigma_i)$ dies after $\delta_j+\epsilon$.
    Since this holds for any \(\epsilon > 0\), we conclude that \(\beta_i^j \leq \delta_j\).

	For the case when $i = j$, apply results in \cref{ss:beta v.s. fillrad} again to get that $\beta_j^j = \delta_j$.
	This completes the proof.
\end{proof}
