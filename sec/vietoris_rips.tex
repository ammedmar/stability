% !TEX root = ../vr_st.tex

\section{Vietoris--Rips persistence}\label{s:preliminaries}


\subsection{Vietoris--Rips complex}\label{ss:vietoris-rips}

We will identify a simplicial complex with its geometric realization.
Let $X$ be a metric space and $r > 0$.
The \defn{Vietoris--Rips complex} $\VR_r(X)$ of $X$ is the simplicial complex whose vertices are the points of $X$ and whose simplices are the finite subsets of $X$ with diameter strictly less than $r$.
Note that if $s \leq t$, then $\VR_s(X)$ is is a subset of $\VR_t(X)$.
So, together with the inclusion maps, $\VR(X)$ defines an $\R$-diagram of spaces with $\VR_r(X) = \emptyset$ for non-positive values of $r$.
We refer to it as the \defn{Vietoris–Rips complex} of $X$.

\subsection{q-tameness results}

\begin{remark}
	In \cite{bib11}, it is proved that if $X$ is a totally bounded metric space, then $\mathrm{PH}_k(\VR_{\bullet}(X); \kk)$ is $q$–tame for any nonnegative integer $k \geq 0$ and any field $\kk$.
\end{remark}

\begin{theorem}[\cite{bib7}]
	If $X$ is a totally bounded metric space, then there is a (unique) persistence barcode associated to $\mathrm{PH}_k(\VR_{\bullet}(X); \kk)$.
\end{theorem}

If $X$ is a totally bounded metric space, then we denote the barcode corresponding to $\mathrm{PH}_k(\VR_{\bullet}(X); \kk)$ by $\barc_{\VR}^k(X; \kk)$.

%The most important existence result for persistence barcodes is Crawley-Boevey’s theorem \cite{bib5} which guarantees the existence of a persistence barcode associated to $V_{\bullet} = (V_r, v_{r,s})$ if $V_{\bullet}$ is pointwise finite-dimensional (i.e., $\dim(V_r) < \infty$ for all $r$). However, for many natural persistence modules (e.g., Vietoris–Rips persistent homology of a nonfinite metric space $X$), it is not straightforward to verify the pointwise finite-dimensionality condition. Nevertheless, in Theorem 2.9, we are able to establish that, if $X$ is totally bounded, then its Vietoris–Rips persistent homology has a (unique) persistence barcode. This is achieved without invoking Crawley-Boevey’s theorem and instead through combining our main (isomorphism) theorem (see Theorem 4.1) with a recent result by Schmahl \cite{bib6}. The proof of Theorem 2.9 can be found in the extended (arXiv) version of this paper \cite{bib7}. The totally boundedness condition is required in the theorem in order to guarantee the following notion of regularity:

\subsection{Filling radius}\label{ss:filling_radius}

Any compact metric space $X$ can be isometrically embedded into the space $L^\infty(X)$ of all bounded real-valued functions on $X$, via the map $x\in X\mapsto d_X(x,\cdot)$ where $d_X(x,\cdot)$ is the distance function to $x$. Let $U_\epsilon(X)$ denote the $\epsilon$-neighborhood of $X$ in $L^\infty(X)$.
For an $n$-dimensional Riemannian manifold $M$, its \defn{filling radius} is defined as the infimal $\epsilon>0$ such that the fundamental class in $\opH_n(M)$ is mapped to zero under the induced map of the inclusion $M \hookrightarrow U_\epsilon(M)$ \cite[page 108]{gromov2007metric}.

\begin{proposition}[{\cite[Prop.~9.4]{lim2020vietoris}}]\label{prop:manifold}
	If $M$ be a closed connected $n$-dimensional Riemannian manifold,
	\[
	(0,2\fillrad{M}) \in \Hbarc{n}{M}.
	\]
\end{proposition}

\subsection{...}\anibal{Where does this statement belong to in the presentation?}

\begin{enumerate}

	\item\label{prop:pH1} {\rm \cite[Prop.~7.10]{virk20201}}.
	If $X$ be a simply-connected geodesic space, then
	\[
	\Hbarc{1}{X} = \emptyset.
	\]
\end{enumerate}

\subsection{Spheres and projective spaces}

For any integer $n \geq 1$ and real number $r > 0$, let $\bbS^n(r)$ be the $n$-sphere of radius $r$, equipped with the geodesic distance.
The \defn{$n$-real projective space} $\rp^n(r)$ is the quotient space $\bbS^n(r)$ by the antipodal map $x \mapsto -x$ for all $x \in \bbS^n$.
Let $[x]$ denote the equivalence class of $x$. We equip $\rp^n(r)$ with the quotient metric defined as
\[
d_{\rp^n(r)}([x],[x'])\defeq\min \{d_{\bbS^n(r)}(x,x'),d_{\bbS^n(r)}(-x,x')\}.
\]
We write $\bbS^n$ and $\rp^n$ for $\bbS^n(1)$ and $\rp^n(2)$ respectively, so that
\[
\diam \left(\bbS^n \right) = \diam\left(\rp^n \right) = \pi.
\]
It is proved in \cite{katz1983filling} that for any $n \geq 1$ their filling radius, as defined in \cref{ss:filling_radius}, satisfy
\[
\fillrad{\bbS^n} = \frac{1}{2}\arccos\left(\frac{-1}{n+1}\right), \qquad \fillrad{\rp^n}=\frac{\pi}{3}.
\]
We denote by $\zeta_n$ twice the filling radius of $\bbS^n$, i.e., $\zeta_n \defeq \arccos\left(\frac{-1}{n+1}\right)$.

In certain intervals, the homotopy type of the Vietoris--Rips complex of these spaces is known.
This information is presented in the following.

\begin{proposition}
	Let $n$ be a positive integer.
	\begin{enumerate}[{\rm (a)}]
		\item\label{prop:S1}{\rm \cite[Thm.~7.4]{adamaszek2017vietoris}.}
		For $t \in \left(\frac{2n\pi}{2n+1}, \frac{2(n+1)\pi}{2n+3}\right]$
		\[
		\VR_t(\bbS^1) \simeq \bbS^{2l+1}.
		\]

		\item\label{prop:Sn}{\rm \cite[Thm.~10]{lim2020vietoris}.}
		For $t \in \left(0, \zeta_n\right]$
		\[
		\VR_t(\bbS^n) \simeq \bbS^n.
		\]

		\item\label{prop:RPn}{\rm \cite[Thm.~4.5]{adams2022metric}.}
		For $t \in \left(0,\frac{2\pi}{3} \right]$
		\[
		\VR_t(\rp^n) \simeq \rp^n.
		\]
	\end{enumerate}
\end{proposition}

\subsubsection{} For the proof of the following theorem, see \cite{bib11} or \cite{bib13, bib14}.

\begin{theorem}
	Let $X$ and $Y$ be compact metric spaces and $\kk$ be an arbitrary field. Then, for any $k \in \mathbb{Z}_{\geq 0}$,
	\[
	d_I \left( \mathrm{PH}_k(\VR_{\bullet}(X); \kk), \mathrm{PH}_k(\VR_{\bullet}(Y); \kk) \right) \leq 2 d_{\mathrm{GH}}(X, Y).
	\]
\end{theorem}

\subsection{Stability of persistent cohomology operations}\label{ss:stability}

Throughout this subsection we consider an arbitrary cohomology operation $\theta$ in $\theta \in \cO(\kk, n; \kk, m)$ for some field $\k$.

%, we establish that the interleaving distance $\di$ between the $\theta$-barcodes of two $\R$-spaces $X$ and $Y$ is bounded above by the homotopy interleaving distance $\dhi$ between $X$ and $Y$ (see Definition \ref{def:dhi}).
%Furthermore, when the $\R$-diagrams are given by the Vietoris--Rips filtrations of two metric spaces, this upper bound can be replaced with twice the Gromov--Hausdorff distance $\dgh$ between the metric spaces. See Theorem \ref{thm:theta stability} for details.

% Let $X$ be a q-tame $\R$-diagram of spaces.
% Recall from \textsection \ref{subsubsec:theta-barcodes} that any cohomology operation $\theta \in \cO(\kk,n; \kk,m)$ defines a morphism of persistence modules $\theta \colon \rH^n(X; \k) \to \rH^m(X; \k)$ for any $\R$-diagram of spaces.
\subsubsection{}\label{lem:di stability}

\lemma Given two $\R$-diagrams of spaces $X$ and $Y$, we have
\[
\di(\thetamodule{X}, \thetamodule{Y}) \leq \di(X,Y).
\]

\begin{proof}
	This follows directly from the fact that $X \mapsto \thetamodule{X}$ defines a functor from the category of spaces to the category of vector spaces over $\k$.
	\anibal{I think more should be said.}
\end{proof}

\subsubsection{}\label{lem:w.h.e. preservance}
\lemma If two $\R$-spaces $X$ and $X'$ are weakly equivalent, then
\[
\di(\thetamodule{X},\thetamodule{X'}) = 0.
\]
%for any cohomology operation $\theta \in \cO(n,m;\k)$ and coefficient field $\k$.

\begin{proof}
	Since $X$ and $X'$ are weakly equivalent, there exists an $\R$-diagram of spaces $Z$ and morphisms $f \colon Z \to X$ and $g \colon Z \to Y$ such that $f_r$ and $g_r$ are weak homotopy equivalences for any $r \in \R$.
	Both $f$ and $g$ induce an isomorphism $f^* \colon \thetamodule{Z} \to \thetamodule{X}$ and $g^* \colon \thetamodule{Z} \to \thetamodule{X'}$.
	From these we have
	\begin{align*}
		\di(\thetamodule{X},\thetamodule{X'}) &\leq
		\di(\thetamodule{X},\thetamodule{Z}) + \di(\thetamodule{Z},\thetamodule{X'}) \\ &=
		0.
	\end{align*}
	%	\ling{to continue: here we need to check that the image of $\theta$ is preserved up to isomorphism under w.h.e.}
	%	\anibal{A bit tricky... One know that a weak equivalence induces iso. in cohomology (Hatcher 4.28), but I am not sure about how to make the inverse consistent across $\R$. This issue should have already appeared the study of persistent homology and the homotopy interleaving distance.}
\end{proof}

\subsubsection{}\label{thm:theta stability}

\theorem Given two $\R$-diagrams of spaces $X$ and $Y$, we have
\begin{equation}\label{eq:theta di-stability}
	\di(\thetamodule{X}, \thetamodule{Y})\leq \dhi(X,Y)
\end{equation}

\begin{proof}
	Let $X$ and $Y$ be $\R$-diagrams of spaces. 
	Take any $\delta > \dhi(X,Y)$.
	By the definition of the homotopy interleaving distance, there exist $\R$-diagrams of spaces $X' \simeq X$ and $Y' \simeq Y$ such that $X'$ and $Y'$ are $\delta$-interleaved.
	By applying the triangle inequality, and Lemmas \ref{lem:w.h.e. preservance} and \ref{lem:di stability}, we obtain
	\begin{align*}
		\di(\thetamodule{X}, \thetamodule{Y}) \leq& \,
		\di(\thetamodule{X}, \thetamodule{X'}) + \di(\thetamodule{X'}, \thetamodule{Y'}) + \di(\thetamodule{Y'}, \thetamodule{Y}) \\ =& \, 
		0 + \di(\thetamodule{X'}, \thetamodule{Y'}) + 0 \\ \leq \,&
		\di(X',Y') \\ \leq \,&
		\delta.
	\end{align*}
	Since $\delta > \dhi(X,Y)$ is arbitrary, we obtain the desired inequality given in Equation (\ref{eq:theta di-stability}).
\end{proof}

\subsubsection{}
\corollary
In the case of the Vietoris--Rips filtration of compact metric spaces $\cX$ and $\cY$, we have
\ling{should we write '$\img$' in the notation of $\theta$-barcodes and $Sq$-barcodes?}
\[
\db(\thetabarc{\cX}, \thetabarc{\cY}) \leq 2 \cdot \dgh(X,Y)
\]

\ling{notation conflict: $X,Y$ are used both for metric spaces and diagrams of spaces. I will use $X$ and $Y$ in this file and we can change them later.}

In the case of the Vietoris--Rips filtration of compact metric spaces $\cX$ and $\cY$, first note that the persistence modules $\thetamodule{\cX}$ and $\thetamodule{\cY}$ are q-tame.
Thus,
\[
\db\big(\thetabarc{\cX}, \thetabarc{\cY}\big) =
\di(\thetamodule{\cX}, \thetamodule{\cY}).
\]
%$\di$ between these two persistence modules agree with the bottleneck distance $\db$ between their corresponding barcodes.
Combining Equation (\ref{eq:theta di-stability}) with Theorem \ref{thm:stability-HI} the stability of $\dhi$, we have
\[
\di(\thetamodule{X}, \thetamodule{Y}) \leq
\dhi(X,Y)\leq 2\cdot \dgh(X,Y).
\]
Therefore, the bottleneck distance $\db$ between the $\theta$-barcodes is also bounded above by $2\cdot \dgh(X,Y)$.

\ling{apply to Steenrod barcodes}