% !TEX root = ../vr_st.tex

\section{Persistent Cohomology operations}\label{s:steenrod}

%\subsection{$K(\pi, n)$ spaces and cohomology operations}

%We review classical material as presented, for example, in \cite{mosheroperations1968}.
%To be consistent with the existing literature we let $\pi$ be an abelian group.
%, not to be confused with homotopy groups denoted $\pi_n$ for $n \geq 0$.
%For the applications of the ideas in this subsection the persistence context $\pi$ will be assumed to be a field.

\subsection{Eilenberg--MacLane spaces}

\subsubsection{}

We denote by $K(\pi, n)$ any cellular space which has only one non-trivial homotopy group, namely, $\pi_n(K(\pi, n)) = \pi$.
Such space is unique up to homotopy, and is referred to as an \defn{Eilenberg--MacLane space}.
For example, since the circle has universal cover the real line and fundamental group isomorphic to $\Z$, it is a model for $K(\Z,1)$.

Let $\cX$ be a space.
The \defn{Hurewicz homomorphism} $h \colon \pi_i(\cX) \to \rH_i(\cX)$ is defined by choosing a generator $u$ of $\rH_i(\bS^i)$ and sending the homotopy class $[f]$ of a based map $f \colon \bS^i \to \cX$ to $f_*(u)$, where $\bS^i$ denotes the $i$-sphere.
Using the properties of this map and the universal coefficient theorem for cohomology one shows the existence of a preferred class $\iota_n \in \rH^n(K(\pi, n); \pi)$ termed the \defn{fundamental class} of $K(\pi, n)$.\footnote{The Hurewicz theorem states that if $\cX$ is such that $\pi_i(\cX)$ is trivial for all $i \leq n$, then the Hurewicz homomorphism is an isomorphism in dimensions $i \leq n$ and a surjection in dimension $n+1$.
	Hence, the first non-trivial homotopy group of $\cX$ and the first non-trivial homology group occur in the same dimension and are isomorphic under $h$.
	For $n=1$, the theorem requires certain modifications which do not affect the following analysis.
	The group $\Hom(\rH_n(\cX), \pi_n(\cX))$ contains $h^{-1}$, the inverse of the Hurewicz homomorphism.
	Subject to the hypothesis on $\cX$, the universal coefficient theorem for homology gives an isomorphism of $\Hom(\rH_n(\cX), \pi_n(\cX))$ and $\rH^n(\cX; \pi_n(\cX))$, and the fundamental class $\iota \in \rH^n(\cX; \pi_n(\cX))$ corresponds to $h^{-1}$ under this isomorphism.}

The relevance of Eilenberg--MacLane space for the study of cohomology is the following bijection:
\[
\begin{tikzcd}[column sep=small, row sep=0]
	{[\cX, K(\pi, n)]} \rar & \rH^n(\cX; \pi) \\
	{[f]} \rar[maps to] & f^*(\iota_n)
\end{tikzcd}
\]
where $[\cX, K(\pi, n)]$ denotes the set of homotopy classes of based maps from $\cX$ to $K(\pi, n)$.
We refer to this bijection as the \defn{representability of cohomology}.

\subsection{Cohomology operations}

\subsubsection{}
Let both $\pi$ and $G$ be abelian groups.
A \defn{cohomology operation} of type $(\pi, n; G, m)$ is a family of functions
\[
\theta_\cX \colon \rH^n(\cX; \pi) \to \rH^m(\cX; G),
\]
one for each space $\cX$ \ling{should this be cellular space?}, satisfying the naturality condition $f^* \theta_{\cY} = \theta_{\cX} f^*$ for any map $f \colon \cX \to \cY$.
We will denote by $\cO(\pi, n; G,m)$ the set of cohomology operations of type $(\pi, n; G,m)$.

\subsubsection{}
There is bijection between $\cO(\pi, n; G,m)$ and homotopy classes of based maps between their Eilenberg--MacLane spaces.
Explicitly, given $[f] \in [K(\pi,n), k(G,m)]$ and an $n$-cohomology class in a space $\cX$ with $\pi$ coefficients, say $g^*(\iota_m)$ for some $g \colon \cX \to K(G, m)$, the value of the cohomology operation $\theta_{\cX}^{[f]}$ on it is
\[
\theta_{\cX}^{[f]}(g^*(\iota_m)) = (f \circ g)^*(\iota_m).
\]
In other words, cohomology operations are parameterized by the cohomology of Eilenberg--MacLane spaces, i.e.,
\[
\cO(\pi, n; G,m) \cong \rH^m(K(\pi,n); G).
\]

\subsection{The Steenrod Algebra}\label{ss:steenrod}

\subsubsection{} The \defn{suspension} of a space $\cX$, denoted as $\sus \cX$, is the product of $\cX$ with the interval $[0,1]$ having each $\cX \times \set{0}$ and $\cX \times \set{1}$ collapsed to a point.
For example, $\sus \bS^n$ is homotopy equivalent to $\bS^{n+1}$.

The inclusion of $\cX$ into $\sus \cX$ induces an isomorphism
\[
\rH^n(\cX) \cong \rH^{n+1}(\sus \cX)
\]
in cohomology with any coefficients, referred to as the \defn{suspension isomorphism}.

\subsubsection{} A \defn{stable cohomology operation} of degree $k$ is a natural family of functions
\[
\set{\theta^n_\cX \colon \rH^n(\cX; \pi) \to \rH^{n+k}(\cX; G)}_{n \in \N}
\]
for each space $\cX$ commuting with suspension isomorphisms, i.e., making the diagram
\[
\begin{tikzcd}
	\rH^n(\cX; \pi) \rar["\theta^n_\cX"] \dar["\cong" left] & \rH^{n+k}(\cX; G) \dar["\cong" right] \\
	\rH^{n+1}(\sus \cX; \pi) \rar["\theta^{n+1}_\cX"] & \rH^{n+1+k}(\sus \cX; G)
\end{tikzcd}
\]
commute for each $n \in \N$.
We will denote by $\cO^{\mathrm{st}}_k(\pi, G)$ the set of stable cohomology operation of degree $k$.

%An example of a cohomology operation that is not stable is the cup-product square
%\[
%\Big\{{\begin{tikzcd}[column sep=small,row sep=-6pt]
%	\rH^n(\cX) \rar & \rH^{2n}(\cX) \\
%	x \rar[maps to] & x^2
%\end{tikzcd}}\Big\}
%\]
%ranging over all $n$.\footnote{To see this notice that suspending this map gives a map $\rH^{n+1}(\Sigma \cX) \to \rH^{2n+1}(\Sigma \cX)$, which has the wrong degree.
	%In fact, on $\rH^*(\Sigma \cX)$, all cup products vanish.}

Using the representability of cohomology, $\cO^{\mathrm{st}}_k(\pi, G)$ can be described as the colimit of the groups
\[
\rH^{n+k}(K(\pi, n); G) \cong [K(\pi, n), K(G, n+k)]
\]
along certain maps
\[
[K(\pi, n), K(\pi, n+k)] \to [K(\pi, n+1), K(G, n+1+k)]
\]
defined by the functoriality of $\sus$ and a natural homotopy equivalence
\begin{equation}\label{eq:suspension of K(pi,n)}
	\sus K(\pi, n) \cong K(\pi, n+1)
\end{equation}
between the suspension of the $n^\th$ Eilenberg--MacLane space and the $(n+1)^\th$.\footnote{\anibal{include spectra?}}
In other words, we have
\[
\cO^\st_k(\pi, G) \cong \colim_n \rH^{n+k}(K(\pi, n); G).
\]

\subsubsection{} We will be interested in combining all stable operations over a fixed coefficient field $\k$ to form a (graded) algebra via compositions:
\[
\cO^\st(\k) = \bigsqcup_k \cO^{\mathrm{st}}_k(\k, \k).
\]

For any prime $p$ let \(\Fp\) be the field with \(p\) elements.
The algebra $\cO^\st(\Fp)$ is denoted $\cA_p$ and referred to as the mod~$p$ \defn{Steenrod Algebra}.
%Its importance in stable homotopy theory is hard to overstate.
%When $\k$ is $\Z/p$ for some prime $p$, the algebra of stable cohomology operations is known as the mod $p$ \defn{Steenrod Algebra} and is of central importance in homotopy theory.
%In the computational part of this paper, we will focus on the mod 2 case, referring to $\cO^{\mathrm{st}}(\Z/2)$ simply as the Steenrod Algebra.

%In terms of the representing object, this is an isomorphism
%\[
%[\cX, K(\pi, n)] \cong [\Sigma \cX, K(\pi, n+1)].
%\]
%The right-hand side is the same as \( [\cX, \Omega K(\pi, n+1)] \), so the suspension isomorphism is induced by the homotopy equivalence \( K(\pi, n) \simeq \Omega K(\pi, n+1) \).
%
%Again using representability, the set of stable cohomology operations will be the colimit of the groups
%\[
%[K(\pi, n), K(\pi, n+k)]
%\]
%along the maps
%\[
%[K(\pi, n), K(\pi, n+k)] \cong [\Omega K(\pi, n+1), \Omega K(\pi, n+k+1)] \cong [\Sigma \Omega K(\pi, n+1), K(\pi, n+k+1)] \to [K(\pi, n+1), K(\pi, n+k+1)].
%\]
%Equivalently, I want the colimit
%\[
%\varinjlim_n \rH^{n+k}(K(\pi, n))
%\]
%along the maps
%\[
%K(\pi, n+1) \to \Sigma \Omega K(\pi, n+1) \simeq \Sigma K(\pi, n).
%\]
%This colimit is the group of degree-\( n \) stable cohomology operations.
%Taking this for all \( n \), we get a graded ring (I can compose two stable cohomology operations to get another one). This ring is the Steenrod Algebra.

\subsubsection{}

The Steenrod algebra $\cA_2$ is generated by the \defn{Steenrod squares} $\set{\Sq^i}_{i \in \N}$, where $\Sq^i \colon \rH^n(\cX; \Ftwo) \to \rH^{n+i}(\cX; \Ftwo)$ for any $n\geq 0$.
These are degree $i$ stable operations which are axiomatically characterized by the following properties holding for all cohomology classes $\alpha,\beta \in \rH^*(\cX; \Ftwo)$ in any space $\cX$:
\begin{enumerate}
	\item \(\Sq^0(\alpha) = \alpha,\)
	\item \(\Sq^i(\alpha) = 0, \quad \text{if } i > \deg\alpha.\)
	\item \(\Sq^i(\alpha) = \alpha^2, \quad \text{if } i = \deg\alpha.\)
	\item \(\Sq^i(\alpha \beta) = \textstyle\sum_{j=0}^{i} \Sq^j(\alpha) \Sq^{i-j}(\beta).\)
\end{enumerate}
We mention that $\Sq^1$ is the Bockstein homomorphism of the exact sequence
\[
0 \to \Z/2 \to \Z/4 \to \Z/2 \to 0.
\]

\subsubsection{}

Consider the system of equatorial inclusions
\[
\bS^0 \subset \bS^1 \subset \bS^2 \subset \dotsb
\]
together with the (compatible) antipodal action on spheres.
A model for \(K(\Z/2, 1)\) is \(\rp^\infty\), the union of the resulting orbits
\[
\rp^0 \subset \rp^1 \subset \rp^2 \subset \dotsb.
\]
%These spaces are of central in the study of the mod~2 Steenrod algebra.

The cohomology algebra of $\rp^n$ with mod 2 coefficients is the polynomial algebra generated by a single element $\sigma$ in degree 1.
That is to say,
\[
\rH^\ast(\rp^\infty; \Ftwo) \cong \Ftwo[\sigma].
\]
Additionally, for any $n \in \N$,
\[
\rH^\ast(\rp^n; \Ftwo) \cong \frac{\Ftwo[\sigma]}{(\sigma^{n+1} = 1)}.
\]

The action of the Steenrod algebra $\cA_2$ on $\rH^*(\rp^n, \Ftwo)$, for $n$ possibly equal to $\infty$, is either the 0 map or is given by
\[
\Sq^k(\sigma^\ell) = \binom{\ell}{k}\sigma^{\ell+k}
\]
for $0 \leq k \leq \frac{n-1}{2}$ and $k \leq \ell$.
Here the binomial coefficient is taken over \(\Ftwo\).

\subsubsection{} For an odd prime \(p\), the Steenrod algebra $\cA_p$ is generated by the \defn{Bockstein} homomorphism of the exact sequence
\[
0 \to \Z/p \to \Z/p^2 \to \Z/p \to 0
\]
and the \defn{Steenrod reduced powers} \(\set{\rP^i}_{i \in \N}\), where $\rP^i \colon \rH^n(\cX; \Fp) \to \rH^{n+2i(p-1)}(\cX; \Fp)$, for any $n\geq 0$.
These are degree \(2i(p-1)\) stable operations which are axiomatically characterized by the following properties holding for all cohomology classes \(\alpha, \beta \in \rH^*(\cX; \Fp)\) in any space \(\cX\):

\begin{enumerate}
	\item \(\rP^0(\alpha) = \alpha,\)
	\item \(\rP^i(\alpha) = 0, \quad \text{if } 2i > \deg\alpha,\)
	\item \(\rP^i(\alpha) = \alpha^p, \quad \text{if } 2i = \deg\alpha,\)
	\item \(\rP^i(\alpha \beta) = \textstyle\sum_{j=0}^{i} \rP^j(\alpha) \rP^{i-j}(\beta).\)
\end{enumerate}

\subsubsection{}

\anibal{Here Lens spaces if possible.}

\subsection{Persistent $\theta$-modules}

\subsubsection{}

A cohomology operation $\theta \in \cO(\kk,m; \kk,n)$ is said to be \defn{linear}, if $\k$ is a field and the map $\theta_{\cX} \colon \rH^m(\cX; \k) \to \rH^n(\cX; \k)$ is linear for any topological space $\cX$.
We denote this subset of cohomology operations by $\cO(m,n;\k)$.
For any prime $p$, the field $\Z/p$ is additively generated, so all cohomology operations in $\cA_p$ are linear.\footnote{An example of an additive map that is not linear is complex conjugation $\bC \to \bC$.}

A linear cohomology operation defines a natural morphism of persistence modules $\theta_X \colon \rH^n(X; \k) \to \rH^m(X; \k)$ for any $\R$-space $X$.
The \defn{image} and \defn{kernel} of $\theta$, denoted $\img\theta$ and $\ker\theta$ respectively, are natural persistence modules defined for any $\R$-space $X$ by
\begin{align*}
	\img(\theta_X)_r &= \img((\theta_X)_r)\,, &
	\ker(\theta_X)_r &= \ker((\theta_X)_r)\,, \\
	\img(\theta_X)_{s,t} &= \rH^n(X;\k)_{s,t}\big|_{\img(\theta_s)}\,, &
	\ker(\theta_X)_{s,t} &= \rH^m(X;\k)_{s,t}\big|_{\ker(\theta_s)}\,,
\end{align*}
the usual notion of image and kernel of a persistence module morphism.

If $X$ is q-tame, the structure maps of $\img\theta$ and $\ker\theta$ are restrictions of structure maps of a q-tame module, therefore $\img(\theta_X)$ and $\ker(\theta_X)$ are q-tame.
In this case we refer to their barcodes as $\theta$-barcodes.
Barcodes of cohomology operations generalize cohomology barcodes since, for the identity and zero maps in $\cO(m,n;\k)$, we have $\img\id_X \cong \ker0_X \cong \rH^n(X;\k)$..

Throughout this subsection we consider an arbitrary cohomology operation $\theta$ in $\theta \in \cO(\kk, n; \kk, m)$ for some field $\k$.

%, we establish that the interleaving distance $\di$ between the $\theta$-barcodes of two $\R$-spaces $X$ and $Y$ is bounded above by the homotopy interleaving distance $\dhi$ between $X$ and $Y$ (see Definition \ref{def:dhi}).
%Furthermore, when the $\R$-diagrams are given by the Vietoris--Rips filtrations of two metric spaces, this upper bound can be replaced with twice the Gromov--Hausdorff distance $\dgh$ between the metric spaces. See Theorem \ref{thm:theta stability} for details.

% Let $X$ be a q-tame $\R$-diagram of spaces.
% Recall from \textsection \ref{subsubsec:theta-barcodes} that any cohomology operation $\theta \in \cO(\kk,n; \kk,m)$ defines a morphism of persistence modules $\theta \colon \rH^n(X; \k) \to \rH^m(X; \k)$ for any $\R$-diagram of spaces.
\subsubsection{}\label{lem:di stability}

\lemma Given two $\R$-diagrams of spaces $X$ and $Y$, we have
\[
\di(\thetamodule{X}, \thetamodule{Y}) \leq \di(X,Y).
\]

\begin{proof}
	This follows directly from the fact that $X \mapsto \thetamodule{X}$ defines a functor from the category of spaces to the category of vector spaces over $\k$.
	\anibal{I think more should be said.}
\end{proof}

\subsubsection{}\label{lem:w.h.e. preservance}
\lemma If two $\R$-spaces $X$ and $X'$ are weakly equivalent, then
\[
\di(\thetamodule{X},\thetamodule{X'}) = 0.
\]
%for any cohomology operation $\theta \in \cO(n,m;\k)$ and coefficient field $\k$.

\begin{proof}
	Since $X$ and $X'$ are weakly equivalent, there exists an $\R$-diagram of spaces $Z$ and morphisms $f \colon Z \to X$ and $g \colon Z \to Y$ such that $f_r$ and $g_r$ are weak homotopy equivalences for any $r \in \R$.
	Both $f$ and $g$ induce an isomorphism $f^* \colon \thetamodule{Z} \to \thetamodule{X}$ and $g^* \colon \thetamodule{Z} \to \thetamodule{X'}$.
	From these we have
	\begin{align*}
		\di(\thetamodule{X},\thetamodule{X'}) &\leq
		\di(\thetamodule{X},\thetamodule{Z}) + \di(\thetamodule{Z},\thetamodule{X'}) \\ &=
		0.
	\end{align*}
	%	\ling{to continue: here we need to check that the image of $\theta$ is preserved up to isomorphism under w.h.e.}
	%	\anibal{A bit tricky... One know that a weak equivalence induces iso. in cohomology (Hatcher 4.28), but I am not sure about how to make the inverse consistent across $\R$. This issue should have already appeared the study of persistent homology and the homotopy interleaving distance.}
\end{proof}

\subsubsection{}\label{thm:theta stability}

\theorem Given two $\R$-diagrams of spaces $X$ and $Y$, we have
\begin{equation}\label{eq:theta di-stability}
	\di(\thetamodule{X}, \thetamodule{Y})\leq \dhi(X,Y)
\end{equation}

\begin{proof}
	Let $X$ and $Y$ be $\R$-diagrams of spaces. 
	Take any $\delta > \dhi(X,Y)$.
	By the definition of the homotopy interleaving distance, there exist $\R$-diagrams of spaces $X' \simeq X$ and $Y' \simeq Y$ such that $X'$ and $Y'$ are $\delta$-interleaved.
	By applying the triangle inequality, and Lemmas \ref{lem:w.h.e. preservance} and \ref{lem:di stability}, we obtain
	\begin{align*}
		\di(\thetamodule{X}, \thetamodule{Y}) \leq& \,
		\di(\thetamodule{X}, \thetamodule{X'}) + \di(\thetamodule{X'}, \thetamodule{Y'}) + \di(\thetamodule{Y'}, \thetamodule{Y}) \\ =& \, 
		0 + \di(\thetamodule{X'}, \thetamodule{Y'}) + 0 \\ \leq \,&
		\di(X',Y') \\ \leq \,&
		\delta.
	\end{align*}
	Since $\delta > \dhi(X,Y)$ is arbitrary, we obtain the desired inequality given in Equation (\ref{eq:theta di-stability}).
\end{proof}

\subsubsection{}
\corollary
In the case of the Vietoris--Rips filtration of compact metric spaces $\cX$ and $\cY$, we have
\ling{should we write '$\img$' in the notation of $\theta$-barcodes and $Sq$-barcodes?}
\[
\db(\thetabarc{\cX}, \thetabarc{\cY}) \leq 2 \cdot \dgh(X,Y)
\]

\ling{notation conflict: $X,Y$ are used both for metric spaces and diagrams of spaces. I will use $X$ and $Y$ in this file and we can change them later.}

In the case of the Vietoris--Rips filtration of compact metric spaces $\cX$ and $\cY$, first note that the persistence modules $\thetamodule{\cX}$ and $\thetamodule{\cY}$ are q-tame.
Thus,
\[
\db\big(\thetabarc{\cX}, \thetabarc{\cY}\big) =
\di(\thetamodule{\cX}, \thetamodule{\cY}).
\]
%$\di$ between these two persistence modules agree with the bottleneck distance $\db$ between their corresponding barcodes.
Combining Equation (\ref{eq:theta di-stability}) with Theorem \ref{thm:stability-HI} the stability of $\dhi$, we have
\[
\di(\thetamodule{X}, \thetamodule{Y}) \leq
\dhi(X,Y)\leq 2\cdot \dgh(X,Y).
\]
Therefore, the bottleneck distance $\db$ between the $\theta$-barcodes is also bounded above by $2\cdot \dgh(X,Y)$.

\ling{apply to Steenrod barcodes}

\subsection{Products}

\subsubsection{}
The \defn{wedge sum} $\cX \vee \cY$ of two pointed spaces is the quotient of their disjoint union by the identification of basepoints.
Given two base point preserving maps $f \colon \cX_1 \to \cX_2$ and $g \colon \cY_1 \to \cY_2$ the map $(f \vee g) \colon \cX_1 \vee \cY_1 \to \cX_2 \vee \cY_2$ is given by
\[
(f \vee g)(z) =
\begin{cases}
	f(z) & z \in \cX_1, \\
	g(z) & z \in \cY_1.\\
\end{cases}
\]
Recall that for ...

\subsubsection{}
For any two pointed $\R$-spaces $X$ and $Y$, the pointed $\R$-space $X \vee Y$, is defined by
\[
(X \vee Y)_r = X_r \vee Y_r \quad\text{and}\quad (X \vee Y)_{s,t} = (X_{s,y} \vee Y_{s,t}).
\]

\medskip\theorem For any linear cohomology operation $\theta$,
\begin{align*}
	\img\theta_{X \vee Y} &\cong \img\theta_X \oplus \img\theta_Y, \\
	\ker\theta_{X \vee Y} &\cong \ker\theta_X \oplus \ker\theta_Y.
\end{align*}

\begin{proof}
%	This follows from naturality and the usual isomorphism $(\iota_\cX \oplus \iota_\cY) \colon \rH^n(\cX \vee \cY; \k) \xra{\cong} \rH^n(\cX; \k) \oplus \rH^n(\cY; \k)$ where $\iota$ denotes the inclusion of a summand into the wedge sum.
\end{proof}

\subsubsection{}

Similarly, for any two $\R$-spaces $X$ and $Y$, the $\R$-space $X \times Y$, is defined by
\[
(X \vee Y)_r = X_r \vee Y_r \quad\text{and}\quad (X \vee Y)_{s,t} = (X_{s,y} \vee Y_{s,t}).
\]

... for Steenrod only ...

\subsubsection{VR version}