% !TEX root = ../vr_st.tex

\section{Persistent Cohomology operations}\label{s:steenrod}

%\subsection{$K(\pi, n)$ spaces and cohomology operations}

%We review classical material as presented, for example, in \cite{mosheroperations1968}.
%To be consistent with the existing literature we let $\pi$ be an abelian group.
%, not to be confused with homotopy groups denoted $\pi_n$ for $n \geq 0$.
%For the applications of the ideas in this subsection the persistence context $\pi$ will be assumed to be a field.

\subsection{$K(\pi, n)$ spaces}

We denote by $K(\pi, n)$ any cellular space which has only one non-trivial homotopy group, namely, $\pi_n(K(\pi, n)) = \pi$.
Such space is unique up to homotopy, and is referred to as an \defn{Eilenberg--MacLane space}.
For example, since the circle has universal cover the real line and fundamental group isomorphic to $\Z$, it is a model for $K(\Z,1)$.

The \defn{Hurewicz homomorphism} $h \colon \pi_i(X) \to \rH_i(X)$ is defined by choosing a generator $u$ of $\rH_i(S^i)$ and sending the homotopy class $[f]$ of a based map $f \colon \bS^i \to X$ to $f^*(u)$.
Using the properties of this map and the universal coefficient theorem for cohomology one shows the existence of a preferred class $\iota_n \in \rH^n(K(\pi, n); \pi)$ termed the \defn{fundamental class} of $K(\pi, n)$.\footnote{The Hurewicz theorem states that if $X$ is such that $\pi_i(X)$ is trivial for all $i \leq n$, then the Hurewicz homomorphism is an isomorphism in dimensions $i \leq n$ and a surjection in dimension $n+1$.
Hence, the first non-trivial homotopy group of $X$ and the first non-trivial homology group occur in the same dimension and are isomorphic under $h$.
For $n=1$, the theorem requires certain modifications which do not affect the following analysis.
The group $\Hom(\rH_n(X), \pi_n(X))$ contains $h^{-1}$, the inverse of the Hurewicz homomorphism.
Subject to the hypothesis on $X$, the universal coefficient theorem for homology gives an isomorphism of $\Hom(\rH_n(X), \pi_n(X))$ and $\rH^n(X; \pi_n(X))$, and the fundamental class $\iota \in \rH^n(X; \pi_n(X))$ corresponds to $h^{-1}$ under this isomorphism.}

The relevance of Eilenberg--MacLane space for the study of cohomology is the following bijection:
\[
\begin{tikzcd}[column sep=small, row sep=0]
	{[X, K(\pi, n)]} \rar & \rH^n(X; \pi) \\
	{[f]} \rar[maps to] & f^*(\iota_n)
\end{tikzcd}
\]
where $[X, K(\pi, n)]$ denotes the set of homotopy classes of based maps from $X$ to $K(\pi, n)$.
We refer to this bijection as the \textit{representability of cohomology}.

\subsection{Cohomology operations}

Let both $\pi$ and $G$ be abelian groups.
A \defn{cohomology operation} of type $(\pi, n; G, m)$ is a family of functions
\[
\theta_X \colon \rH^n(X; \pi) \to \rH^m(X; G),
\]
one for each space $X$, satisfying the naturality condition $f^* \theta = \theta f^*$ for any map $f \colon X \to Y$.
We will denote by $\cO(\pi, n; G,m)$ the set of cohomology operations of type $(\pi, n; G,m)$.

There is bijection between $\cO(\pi, n; G,m)$ and homotopy classes of based maps between their Eilenberg--MacLane spaces.
Explicitly, given $[f] \in [K(\pi,n), k(G,m)]$ and an $n$-cohomology class in a space $X$ with $\pi$ coefficients, say $g^*(\iota_m)$ for some $g \colon X \to K(G, m)$, the value of the cohomology operation $\theta_{[f]}$ on it is
\[
\theta_{[f]}(g^*(\iota_m)) = (f \circ g)^*(\iota_m).
\]
In other words, cohomology operations are parameterized by the cohomology of Eilenberg--MacLane spaces, i.e.,
\[
\cO(\pi, n; G,m) \cong \rH^m(K(\pi,n); G).
\]

\subsection{The Steenrod Algebra}\label{ss:steenrod}

\subsubsection{} For a space $X$ let $\sus X$ denote the product of $X$ with the interval $[0,1]$ having each $X \times \set{0}$ and $X \times \set{1}$ collapsed to a point.
For example, $\sus \bS^n \sim \bS^{n+1}$.

The inclusion of $X$ into $\sus X$ induces an isomorphism
\[
\rH^n(X) \cong \rH^{n+1}(\sus X)
\]
in cohomology with any coefficients, referred to as the \defn{suspension isomorphism}.

\subsubsection{} A \defn{stable cohomology operation} of degree $k$ is a natural family of functions
\[
\set{\theta^n_X \colon \rH^n(X; \pi) \to \rH^{n+k}(X; G)}_{n \in \N}
\]
for each space $X$ commuting with suspension isomorphisms, i.e., making the diagram
\[
\begin{tikzcd}
	\rH^n(X; \pi) \rar["\theta^n_X"] \dar & \rH^{n+k}(X; G) \dar \\
	\rH^{n+1}(\sus X; \pi) \rar["\theta^{n+1}_X"] & \rH^{n+1+k}(\sus X; G)
\end{tikzcd}
\]
commute for each $n \in \N$.
We will denote by $\cO^{\mathrm{st}}_k(\pi, G)$ the set of stable cohomology operation of degree $k$.

%An example of a cohomology operation that is not stable is the cup-product square
%\[
%\Big\{{\begin{tikzcd}[column sep=small,row sep=-6pt]
%	\rH^n(X) \rar & \rH^{2n}(X) \\
%	x \rar[maps to] & x^2
%\end{tikzcd}}\Big\}
%\]
%ranging over all $n$.\footnote{To see this notice that suspending this map gives a map $\rH^{n+1}(\Sigma X) \to \rH^{2n+1}(\Sigma X)$, which has the wrong degree.
%In fact, on $\rH^*(\Sigma X)$, all cup products vanish.}

Using the representability of cohomology, $\cO^{\mathrm{st}}_k(\pi, G)$ can be described as the colimit of the groups
\[
H^{n+k}(K(\pi, n); G) \cong [K(\pi, n), K(G, n+k)]
\]
along certain maps
\[
[K(\pi, n), K(\pi, n+k)] \to [K(\pi, n+1), K(G, n+1+k)]
\]
defined by the functoriality of $\sus$ and a natural homotopy equivalence
\begin{equation}\label{eq:suspension of K(pi,n)}
	\sus K(\pi, n) \cong K(\pi, n+1)
\end{equation}
between the suspension of the $n^\th$ Eilenberg--MacLane space and the $(n+1)^\th$.\footnote{\anibal{include spectra?}}
In other words, we have
\[
\cO^\st_k(\pi, G) \cong \colim_n \rH^{n+k}(K(\pi, n); G).
\]

\subsubsection{} We will be interested in combining all stable operations
\[
\cO^\st(\k) = \bigsqcup_k \cO^{\mathrm{st}}_k(\k, \k),
\]
over a fixed coefficient field $\k$.
With compositions, $\cO^\st(\k)$ forms a (graded) algebra.
For any prime $p$, the algebra $\cO^\st(\Z/p)$ is denoted $\cA_p$ and referred to as the mod~$p$ \defn{Steenrod Algebra}.
%Its importance in stable homotopy theory is hard to overstate.
%When $\k$ is $\Z/p$ for some prime $p$, the algebra of stable cohomology operations is known as the mod $p$ \defn{Steenrod Algebra} and is of central importance in homotopy theory.
%In the computational part of this paper, we will focus on the mod 2 case, referring to $\cO^{\mathrm{st}}(\Z/2)$ simply as the Steenrod Algebra.

%In terms of the representing object, this is an isomorphism
%\[
%[X, K(\pi, n)] \cong [\Sigma X, K(\pi, n+1)].
%\]
%The right-hand side is the same as \( [X, \Omega K(\pi, n+1)] \), so the suspension isomorphism is induced by the homotopy equivalence \( K(\pi, n) \simeq \Omega K(\pi, n+1) \).
%
%Again using representability, the set of stable cohomology operations will be the colimit of the groups
%\[
%[K(\pi, n), K(\pi, n+k)]
%\]
%along the maps
%\[
%[K(\pi, n), K(\pi, n+k)] \cong [\Omega K(\pi, n+1), \Omega K(\pi, n+k+1)] \cong [\Sigma \Omega K(\pi, n+1), K(\pi, n+k+1)] \to [K(\pi, n+1), K(\pi, n+k+1)].
%\]
%Equivalently, I want the colimit
%\[
%\varinjlim_n \rH^{n+k}(K(\pi, n))
%\]
%along the maps
%\[
%K(\pi, n+1) \to \Sigma \Omega K(\pi, n+1) \simeq \Sigma K(\pi, n).
%\]
%This colimit is the group of degree-\( n \) stable cohomology operations.
%Taking this for all \( n \), we get a graded ring (I can compose two stable cohomology operations to get another one). This ring is the Steenrod Algebra.

\subsubsection{} 
The Steenrod algebra $\cA_2$ is generated by the \defn{Steenrod squares} $\set{\Sq^i}_{i \in \N}$.
These are degree $i$ stable operations which are axiomatically characterized by the following properties holding for all cohomology classes $\alpha,\beta \in \rH^*(\cX; \Z/2)$ in any space $\cX$:
\begin{enumerate}
	\item \(\Sq^0(\alpha) = \alpha.\)
	\item \(\Sq^i(\alpha) = 0, \quad \text{if } i > \deg\alpha.\)
	\item \(\Sq^i(\alpha) = \alpha^2, \quad \text{if } i = \deg\alpha.\)
	\item \(\Sq^i(\alpha \beta) = \textstyle\sum_{j=0}^{i} \Sq^j(\alpha) \Sq^{i-j}(\beta).\)
\end{enumerate}
We mention that $\Sq^1$ is the Bockstein $\beta$ of the exact sequence
\[
0 \to \Z/2 \to \Z/4 \to \Z/2 \to 0.
\]

\subsubsection{} For an odd prime \(p\), the Steenrod algebra $\cA_p$ is generated by the \defn{Bockstein} \(\beta\) of the exact sequence
\[
0 \to \Z/p \to \Z/p^2 \to \Z/p \to 0
\]
and the \defn{Steenrod reduced powers} \(\set{\mathcal{P}^i}_{i \in \N}\).
These are degree \(2i(p-1)\) stable operations which are axiomatically characterized by the following properties holding for all cohomology classes \(\alpha, \beta \in \rH^*(\cX; \Z/p)\) in any space \(\cX\):

\begin{enumerate}
	\item \(\mathcal{P}^0(\alpha) = \alpha.\)
	\item \(\mathcal{P}^i(\alpha) = 0, \quad \text{if } 2i > \deg\alpha.\)
	\item \(\mathcal{P}^i(\alpha) = \alpha^p, \quad \text{if } 2i = \deg\alpha.\)
	\item \(\mathcal{P}^i(\alpha \beta) = \textstyle\sum_{j=0}^{i} \mathcal{P}^j(\alpha) \mathcal{P}^{i-j}(\beta).\)
\end{enumerate}

\subsection{$\theta$-barcodes}

\subsubsection{} Let $\varphi \colon V \to W$ be a morphisms between q-tame persistence modules.
The \defn{image persistent module} $\img(\varphi)$ is defined by
\[
\img(\varphi)_r = \img(\varphi_r)
\qquad\text{and}\qquad
\img(\varphi)_{s,t} = W_{s,t}\big|_{\img(\varphi_s)}\,.
\]
Since the structure maps of $\img \varphi$ are restrictions of structure maps of a q-tame module, the persistence module $\img(\varphi)$ is q-tame.
A similar construction defines the \defn{kernel persistence module} $\ker(\varphi)$.

\subsubsection{} \label{subsubsec:theta-barcodes}
Any cohomology operation $\theta \in \cO(\kk,n; \kk,m)$ defines a morphism of persistence modules $\theta_X \colon \rH^n(X; \k) \to \rH^m(X; \k)$ for any $\R$-diagram of spaces $X$.
If $X$ is q-tame, we define its \defn{$\theta$-barcode} as the barcode of $\img\theta_X$.

For example, taking $\theta$ to be the identity map in degree $n$, the $\theta$-barcode of $X$ is equal to the barcode of the persistent $n$-cohomology of $X$.

%\subsubsection{} $\Sq^k$-barcodes were introduced in \cite{medina2022per_st} together with an implementation for their computation for finite filtered simplicial complexes.

\subsection{Stability of $\theta$-barcodes}\label{ss:stability}

Throughout this subsection we consider an arbitrary cohomology operation $\theta$ in $\theta \in \cO(\kk, n; \kk, m)$ for some field $\k$.

%, we establish that the interleaving distance $\di$ between the $\theta$-barcodes of two $\R$-spaces $X$ and $Y$ is bounded above by the homotopy interleaving distance $\dhi$ between $X$ and $Y$ (see Definition \ref{def:dhi}).
%Furthermore, when the $\R$-diagrams are given by the Vietoris--Rips filtrations of two metric spaces, this upper bound can be replaced with twice the Gromov--Hausdorff distance $\dgh$ between the metric spaces. See Theorem \ref{thm:theta stability} for details.

% Let $X$ be a q-tame $\R$-diagram of spaces.
% Recall from \textsection \ref{subsubsec:theta-barcodes} that any cohomology operation $\theta \in \cO(\kk,n; \kk,m)$ defines a morphism of persistence modules $\theta \colon \rH^n(X; \k) \to \rH^m(X; \k)$ for any $\R$-diagram of spaces.
\subsubsection{}\label{lem:di stability}

\lemma Given two $\R$-diagrams of spaces $X$ and $Y$, we have
\[
\di(\thetamodule{X}, \thetamodule{Y}) \leq \di(X,Y).
\]

\begin{proof}
	This follows directly from the fact that $X \mapsto \thetamodule{X}$ defines a functor from the category of spaces to the category of vector spaces over $\k$.
	\anibal{I think more should be said.}
\end{proof}

\subsubsection{}\label{lem:w.h.e. preservance}
\lemma If two $\R$-spaces $X$ and $X'$ are weakly equivalent, then
\[
\di(\thetamodule{X},\thetamodule{X'}) = 0.
\]
%for any cohomology operation $\theta \in \cO(n,m;\k)$ and coefficient field $\k$.

\begin{proof}
	Since $X$ and $X'$ are weakly equivalent, there exists an $\R$-diagram of spaces $Z$ and morphisms $f \colon Z \to X$ and $g \colon Z \to Y$ such that $f_r$ and $g_r$ are weak homotopy equivalences for any $r \in \R$.
	Both $f$ and $g$ induce an isomorphism $f^* \colon \thetamodule{Z} \to \thetamodule{X}$ and $g^* \colon \thetamodule{Z} \to \thetamodule{X'}$.
	From these we have
	\begin{align*}
		\di(\thetamodule{X},\thetamodule{X'}) &\leq
		\di(\thetamodule{X},\thetamodule{Z}) + \di(\thetamodule{Z},\thetamodule{X'}) \\ &=
		0.
	\end{align*}
	%	\ling{to continue: here we need to check that the image of $\theta$ is preserved up to isomorphism under w.h.e.}
	%	\anibal{A bit tricky... One know that a weak equivalence induces iso. in cohomology (Hatcher 4.28), but I am not sure about how to make the inverse consistent across $\R$. This issue should have already appeared the study of persistent homology and the homotopy interleaving distance.}
\end{proof}

\subsubsection{}\label{thm:theta stability}

\theorem Given two $\R$-diagrams of spaces $X$ and $Y$, we have
\begin{equation}\label{eq:theta di-stability}
	\di(\thetamodule{X}, \thetamodule{Y})\leq \dhi(X,Y)
\end{equation}

\begin{proof}
	Let $X$ and $Y$ be $\R$-diagrams of spaces. 
	Take any $\delta > \dhi(X,Y)$.
	By the definition of the homotopy interleaving distance, there exist $\R$-diagrams of spaces $X' \simeq X$ and $Y' \simeq Y$ such that $X'$ and $Y'$ are $\delta$-interleaved.
	By applying the triangle inequality, and Lemmas \ref{lem:w.h.e. preservance} and \ref{lem:di stability}, we obtain
	\begin{align*}
		\di(\thetamodule{X}, \thetamodule{Y}) \leq& \,
		\di(\thetamodule{X}, \thetamodule{X'}) + \di(\thetamodule{X'}, \thetamodule{Y'}) + \di(\thetamodule{Y'}, \thetamodule{Y}) \\ =& \, 
		0 + \di(\thetamodule{X'}, \thetamodule{Y'}) + 0 \\ \leq \,&
		\di(X',Y') \\ \leq \,&
		\delta.
	\end{align*}
	Since $\delta > \dhi(X,Y)$ is arbitrary, we obtain the desired inequality given in Equation (\ref{eq:theta di-stability}).
\end{proof}

\subsubsection{}
\corollary
In the case of the Vietoris--Rips filtration of compact metric spaces $\cX$ and $\cY$, we have
\ling{should we write '$\img$' in the notation of $\theta$-barcodes and $Sq$-barcodes?}
\[
\db(\thetabarc{\cX}, \thetabarc{\cY}) \leq 2 \cdot \dgh(X,Y)
\]

\ling{notation conflict: $X,Y$ are used both for metric spaces and diagrams of spaces. I will use $X$ and $Y$ in this file and we can change them later.}

In the case of the Vietoris--Rips filtration of compact metric spaces $\cX$ and $\cY$, first note that the persistence modules $\thetamodule{\cX}$ and $\thetamodule{\cY}$ are q-tame.
Thus,
\[
\db\big(\thetabarc{\cX}, \thetabarc{\cY}\big) =
\di(\thetamodule{\cX}, \thetamodule{\cY}).
\]
%$\di$ between these two persistence modules agree with the bottleneck distance $\db$ between their corresponding barcodes.
Combining Equation (\ref{eq:theta di-stability}) with Theorem \ref{thm:stability-HI} the stability of $\dhi$, we have
\[
\di(\thetamodule{X}, \thetamodule{Y}) \leq
\dhi(X,Y)\leq 2\cdot \dgh(X,Y).
\]
Therefore, the bottleneck distance $\db$ between the $\theta$-barcodes is also bounded above by $2\cdot \dgh(X,Y)$.

\ling{apply to Steenrod barcodes}