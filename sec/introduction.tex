% !TEX root = ../vr_st.tex

\section{Introduction} \label{s:introduction}

When studying the shape and structure of spaces equipped with a filtration, \textit{persistent homology} has emerged as a crucial tool in both applied and theoretical topology.
By tracking the evolution of homological features across the filtration, this technique has found widespread applications in fields ranging from data analysis and machine learning to simplectic geometry and functional analysis.
A few among many examples can be found in: ... \ling{to fill in}
Despite its successes, persistent homology has limitations that mirror those of classical homology in the study of unfiltered topological spaces.
Just as homology fails to distinguish spaces with different homotopy types, persistent homology can overlook significant structural aspects of their filtered counterparts.

To address these shortcomings, researchers have begun exploring extensions of persistent homology that incorporate more refined homotopical information.
For example, the cup product in cohomology has been studied in \cite{contessoto_et_al:LIPIcs.SoCG.2022.31, memoli2024persistent, huang2005cup, yarmola2010persistence, herscovich2018higher, belchi2021a, contreras2022persistent}.
We focus on cohomology operations, which are natural transformations between cohomology functors.
%It had long been envisioned that these would enhance persistent topology.
%, as explored, for instance, in Aubrey's Ph.D. thesis \cite{aubrey2011thesis} \anibal{anyone else?}.
The first notable advances in this direction were achieved in \cite{medina2022per_st}, which, using new methods for the computation of Steenrod squares (\cite{medina2023fast_sq}), proposed and implemented algorithms for computing \textit{Steenrod barcodes} of finite filtered simplicial complexes.
The resulting tool, \href{https://steenroder.github.io/steenroder/}{\texttt{steenroder}}, enabled the detection of nontrivial Steenrod barcodes in a sample of the conformation space of the cyclo-octane molecule.

Despite the conceptual and computational advances of \cite{medina2022per_st}, a comprehensive exploration of persistent cohomology operations has remained absent, leaving key questions about their theoretical properties and broader applicability unanswered.
In this paper, we systematically treat \textit{persistent cohomology operations} providing them with a general and firm foundation, and present applications of these invariants to Riemannian geometry.
We now summarize our contributions.

Let \(\k\) be a field.
A \textit{\(\k\)-linear cohomology operation} \(\theta\) is a natural transformation
\[
\theta \colon \rH^\ell(-;\k) \to \rH^\degp(-;\k),
\]
between cohomology functors.
The image and kernel of \(\theta\) define two persistent modules for every persistent space \(X\) which we denote \(\img_\theta(X)\) and \(\ker_\theta(X)\).
When \(X\) is the Vietoris--Rips filtration \(\VR(\cX)\) of a metric space \(\cX\) we simplify notation writing \(\img_\theta^\VR(\cX)\) and \(\ker_\theta^\VR(\cX)\).
We now state part of two decomposition theorems fully stated and proven in \cref{ss:sums_products}.

\medskip\theorem For any two pointed metric spaces $\cX$ and $\cY$ and linear cohomology operation \(\theta\) one has:
\[
\img_\theta^\VR(\cX \vee \cY) \cong \img_\theta^\VR(\cX) \oplus \img_\theta^\VR(\cY),
\]
where the wedge sum \(\cX \vee \cY\) is equipped with the gluing metric.

\medskip\theorem
For any two metric spaces $\cX$ and $\cY$, the total Steenrod square satisfies:
\[
\img^{\VR}_\Sq(\cX \times \cY) \cong \img^{\VR}_\Sq(\cX) \ot \img^{\VR}_\Sq(\cY),
\]
where the product \(\cX \times \cY\) is equipped with the \(\ell^\infty\) metric.

STABILITY

SHARPER ESTIMATES

%Furthermore, our study offers a framework for leveraging cohomology operations to achieve finer discriminatory power of filtered spaces.
%Specifically, we provide tighter bounds for the Gromov--Hausdorff distance of metric spaces than those obtained using persistent homology alone.
%In particular, we explore the implications of persistent cohomology operations for the study of metric spaces when combined with the \textit{Vietoris--Rips construction}.

%Given a
%
%EXPAND One of the most useful features of persistent homology is their stability under perturbations of the input filtered space.
%Stability i	s a cornerstone of persistent topology; it guarantees that small changes in the input (e.g., noise in data or slight perturbations of a metric space) result in correspondingly small changes in the persistence invariants.
%We extend the notion of stability to persistent cohomology operations, providing rigorous estimates that demonstrate their robustness.
%
%EXPAND Additionally, we derive explicit \textit{product} and \textit{wedge sum formulas} that describe how these invariants decompose when applied to combinations of filtered spaces.
%These results significantly extend the algebraic toolkit available for analyzing filtered topological spaces through persistence.

%Armed with these theoretical developments, we turn our attention to applications in the geometry of metric spaces.
%Using our framework, we derive \textit{Gromov--Hausdorff estimates} that illustrate the enhanced discriminatory power of persistent cohomology operations.
%The Gromov--Hausdorff distance is a fundamental metric in geometry that measures how close two metric spaces are to being isometric.
%By applying persistent cohomology operations to the Vietoris--Rips complexes of two metric spaces --a wedge sum of spheres and the corresponding real projective space-- we show that these invariants can provide tighter lower bounds for Gromov--Hausdorff distance than those obtained from standard persistent homology.

%Building on these theoretical developments, we then explore applications in the geometry of metric spaces.
%In particular, we derive refined \textit{Gromov--Hausdorff estimates} that demonstrate the enhanced discriminatory power of persistent cohomology operations.
%The Gromov--Hausdorff distance serves as a fundamental metric in geometry, quantifying how closely two metric spaces approximate each other.
%By applying persistent cohomology operations to the Vietoris--Rips complexes of two specific metric spaces --namely, a wedge sum of spheres and a real projective space-- we show that these invariants yield tighter lower bounds for the Gromov--Hausdorff distance than those achievable through standard persistent homology.

%Overall, this work makes several key contributions.
%First, we provide a systematic study of persistent cohomology operations, developing a theoretical framework that encompasses their stability and product structures.
%Second, we demonstrate how these operations can be used to derive tighter Gromov--Hausdorff estimates, highlighting their potential for applications in metric geometry and data analysis.
%Through these investigations, we aim to broaden the scope of persistence theory, opening new avenues for the study of filtered spaces and their topological properties.

\subsection*{Previous work}


