% !TEX root = ../vr_st.tex

\section{Introduction} \label{s:introduction}

When studying the shape and structure of spaces equipped with a filtration, \textit{persistent homology} has emerged as a crucial tool in both applied and theoretical topology.
By tracking the evolution of homological features across the filtration, this technique has found widespread applications in fields ranging from data analysis and machine learning to simplectic geometry and functional analysis.
A few among many examples can be found in: ...
Despite its successes, persistent homology has limitations that mirror those of classical homology in the study of unfiltered topological spaces.
Just as homology fails to distinguish spaces with different homotopy types, persistent homology can overlook significant structural aspects of their filtered counterparts.

To address these shortcomings, researchers have begun exploring extensions of persistent homology that incorporate more refined homotopical information.
For example, the cup product in cohomology has been studied in \cite{contessoto_et_al:LIPIcs.SoCG.2022.31, memoli2024persistent, huang2005cup, yarmola2010persistence, herscovich2018higher, belchi2021a, contreras2022persistent}.
We focus on cohomology operations, which are natural transformations between cohomology functors.
%It had long been envisioned that these would enhance persistent topology.
%, as explored, for instance, in Aubrey's Ph.D. thesis \cite{aubrey2011thesis} \anibal{anyone else?}.
\lz{Why is \cite{aubrey2011thesis} removed?}
The first notable advances in this direction were achieved in \cite{medina2022per_st}, which, using new methods for the computation of Steenrod squares (\cite{medina2023fast_sq}), proposed and implemented algorithms for computing the mod~2 \textit{Steenrod barcodes} of finite filtered simplicial complexes.
The resulting tool, \href{https://steenroder.github.io/steenroder/}{\texttt{steenroder}}, enabled the detection of nontrivial Steenrod barcodes on a point cloud sampled from the conformation space of the cyclo-octane molecule.

Despite the conceptual and computational advances of \cite{medina2022per_st}, a comprehensive exploration of persistent cohomology operations has remained absent, leaving key questions about their theoretical properties and broader applicability unanswered.
In this paper, we systematically treat \textit{persistent cohomology operations} providing them with a general and firm foundation, and present applications of these invariants to Riemannian geometry.

Let \(\k\) be a field.
A \textit{\(\k\)-linear cohomology operation} \(\theta\) is a natural transformation
\[
\theta \colon \rH^\ell(-;\k) \to \rH^\degp(-;\k),
\]
between cohomology functors.
From now on we will omit \(\k\) from the notation.

Given a functor \(X\) from the poset category $\R$ to the category of cellular spaces, the image and kernel of \(\theta\) define two persistent modules, denoted \(\img_\theta(X)\) and \(\ker_\theta(X)\), which we collectively refer to as \textit{persistent \(\theta\)-modules}.
When \(X\) is the Vietoris--Rips filtration \(\VR(\cX)\) of a metric space \(\cX\), we use the simplified notation \(\img_\theta^\VR(\cX)\) and \(\ker_\theta^\VR(\cX)\).

Our first contributions, fully detailed and proven in \cref{ss:sums_products}, presents decomposition theorems for wedge sums and products.
Here we state them only for \(\img^\VR_\theta\), mentioning that the same holds for \(\ker^\VR_\theta\).

\medskip\theorem For any two pointed metric spaces $\cX$ and $\cY$, a linear cohomology operation \(\theta\) satisfies:
\[
\img_\theta^\VR(\cX \vee \cY) \cong \img_\theta^\VR(\cX) \oplus \img_\theta^\VR(\cY),
\]
where the wedge sum \(\cX \vee \cY\) is equipped with the gluing metric.

\medskip\theorem
For any two metric spaces $\cX$ and $\cY$, the total Steenrod reduced power operation over any prime satisfies:
\[
\img^{\VR}_\rP(\cX \times \cY) \cong \img^{\VR}_\rP(\cX) \ot \img^{\VR}_\rP(\cY),
\]
where the product \(\cX \times \cY\) is equipped with the \(\ell^\infty\) metric.

\medskip One of the primary reasons persistent homology is widely used is its stability.
Persistent cohomology operations share this stability, as stated below for \(\img^\VR_\theta\).
Once again we omit the version for \(\ker^\VR_\theta\), referring the reader to \cref{ss:stability} for a complete treatment.

\medskip\theorem For any two metric spaces $\cX$ and $\cY$ and a linear cohomology operation~$\theta$:
\[
\di\big(\img_\theta^\VR(\cX),\, \img_\theta^\VR(\cY)\big) \leq 2 \cdot \dgh(\cX,\cY),
\]
where \(\di\) and \(\dgh\) are respectively the interleaving and Gromov--Hausdorff distances.

\medskip In recent work, \cite{lim2024vietoris} established a deep connection between the Vietoris--Rips filtration of a metric space \(\cX\) and its Kuratowski embedding \(K \colon \cX \to \rL^\infty(\cX)\), the canonical isometric inclusion of \(\cX\) into a Banach space.
More precisely, \cite{lim2024vietoris} proved that if \(\rU_r(\cX)\) is the radius \(r\) neighborhood of \(K(\cX)\), then \(\VR_{2r}(\cX)\) and \(\rU_r(\cX)\) are naturally homotopy equivalent for every \(r > 0\).

Gromov's filling radius is defined as the smallest \(r\) for which the fundamental class of a closed connected Riemannian manifold becomes null-homologous in \(\rU_r(\cM)\).
Using this invariant and other similar critical radii, we construct pairs of Riemannian pseudomanifolds for which the Gromov--Hausdorff estimates derived from the stability of persistent cohomology operations provide sharper bounds than those obtained from persistent homology.

%Using this invariant and other similar critical radii, we can construct examples of pairs of Riemannian pseudomanifolds where the Gromov--Hausdorff estimate derived from the stability of persistent cohomology operations, provides a sharper bound than those obtained through persistent homology.

%Specifically, we consider in one hand the wedge of round spheres \(\VS = \bS^1 \vee \dots \vee \bS^n\), each having diameter \(\pi\), and in the other, the real projective space \(\rp^n\), also with diameter \(\pi\), obtained as the quotient of a round sphere under the antipodal action.

Specifically, we consider, on one hand, the real projective space \(\rp^n\), with diameter \(\pi\), obtained as the quotient of a round sphere under the antipodal action; and on the other, the wedge sum of round spheres \(\bS_{\mathbb{RP}^n} = \bS^1 \vee \dots \vee \bS^n\), each with diameter \(\pi\).

\medskip\theorem
For every \(n > 1\) we have the following inequalities:
\begin{enumerate}
	\item For any \(m \in \N\)
	\[
	\di\Big(\rH^\VR_m(\rp^n),\, \rH^\VR_m(\bS_{\rp^n})\Big) < \frac{\pi}{4}.
	\]

	\item There are \(\degp, k \in \N\) such that
	\[
	\di\Big(\img_{\Sq^k}^\VR(\rp^n),\, \img_{\Sq^k}^\VR(\bS_{\rp^n})\Big) \geq \frac{\pi}{3},
	\]
	where \(\Sq^k\) is the Steenrod square to degree \(\degp\).
\end{enumerate}

\medskip As explained in \cref{ss:genberal_distance_comparison}, our arguments would also apply to Lens spaces and odd prime Steenrod operations if certain statements about their critical radii were known, which we leave for future work.