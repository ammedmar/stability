% !TEX root = ../vr_st.tex

\section{Introduction} \label{s:introduction}

When studying the shape and structure of spaces equipped with a filtration, \textit{persistent homology} has emerged as a crucial tool in both applied and theoretical topology.
By tracking the evolution of homological features across the filtration, this technique has found widespread applications in fields ranging from data analysis and machine learning to simplectic geometry and functional analysis.
A few among many examples can be found in: ...
Despite its successes, persistent homology has limitations that mirror those of classical homology in the study of unfiltered topological spaces.
Just as homology fails to distinguish spaces with different homotopy types, persistent homology can overlook significant structural aspects of their filtered counterparts.

To address these shortcomings, researchers have begun exploring extensions of persistent homology that incorporate more refined homotopical information. One method is the shift to cohomology, which carries more algebraic structure.
For the incorporation of the cup product into the persistent pipeline we refer to \cite{memoli2024persistent} and the references therein.\anibal{Please complete list}
The approach presented here is grounded in cohomology operations, such as the Steenrod squares, which are powerful yet computable homotopical invariants.
Previous efforts to adapt Steenrod squares to the realm of persistent topology include Aubrey's PhD thesis \cite{aubrey2011thesis} and the work of Lupo et al.\ \cite{medina2022per_st}.
The latter introduces computable invariants, known as Steenrod barcodes, along with software for its application to real-world data.
Despite these advances, a comprehensive exploration of persistent cohomology operations remained absent, leaving key questions about their properties and broader applications in theoretical mathematics unresolved.

In this paper, we investigate \textit{persistent cohomology operations} with the goal of filling this gap.
Specifically, we aim to systematically define these, explore their stability properties, and derive decomposition formulas for their associated barcodes.
Furthermore, our study offers a framework for leveraging cohomology operations to achieve finer discriminatory power of filtered spaces.
Specifically, we provide tighter bounds for the Gromov--Hausdorff distance of metric spaces than those obtained using persistent homology alone.
%In particular, we explore the implications of persistent cohomology operations for the study of metric spaces when combined with the \textit{Vietoris--Rips construction}.

%Given a
%
%EXPAND One of the most useful features of persistent homology is their stability under perturbations of the input filtered space.
%Stability i	s a cornerstone of persistent topology; it guarantees that small changes in the input (e.g., noise in data or slight perturbations of a metric space) result in correspondingly small changes in the persistence invariants.
%We extend the notion of stability to persistent cohomology operations, providing rigorous estimates that demonstrate their robustness.
%
%EXPAND Additionally, we derive explicit \textit{product} and \textit{wedge sum formulas} that describe how these invariants decompose when applied to combinations of filtered spaces.
%These results significantly extend the algebraic toolkit available for analyzing filtered topological spaces through persistence.

%Armed with these theoretical developments, we turn our attention to applications in the geometry of metric spaces.
%Using our framework, we derive \textit{Gromov--Hausdorff estimates} that illustrate the enhanced discriminatory power of persistent cohomology operations.
%The Gromov--Hausdorff distance is a fundamental metric in geometry that measures how close two metric spaces are to being isometric.
%By applying persistent cohomology operations to the Vietoris--Rips complexes of two metric spaces --a wedge sum of spheres and the corresponding real projective space-- we show that these invariants can provide tighter lower bounds for Gromov--Hausdorff distance than those obtained from standard persistent homology.

%Building on these theoretical developments, we then explore applications in the geometry of metric spaces.
%In particular, we derive refined \textit{Gromov--Hausdorff estimates} that demonstrate the enhanced discriminatory power of persistent cohomology operations.
%The Gromov--Hausdorff distance serves as a fundamental metric in geometry, quantifying how closely two metric spaces approximate each other.
%By applying persistent cohomology operations to the Vietoris--Rips complexes of two specific metric spaces --namely, a wedge sum of spheres and a real projective space-- we show that these invariants yield tighter lower bounds for the Gromov--Hausdorff distance than those achievable through standard persistent homology.

%Overall, this work makes several key contributions.
%First, we provide a systematic study of persistent cohomology operations, developing a theoretical framework that encompasses their stability and product structures.
%Second, we demonstrate how these operations can be used to derive tighter Gromov--Hausdorff estimates, highlighting their potential for applications in metric geometry and data analysis.
%Through these investigations, we aim to broaden the scope of persistence theory, opening new avenues for the study of filtered spaces and their topological properties.

\subsection*{Previous work}


