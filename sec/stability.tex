\subsection{Stability}\label{ss:stability}

%Throughout this subsection we consider cohomology operations in $\cO(\k, m; \k, n)$ for some arbitrary $m$ and $n$.

%, we establish that the interleaving distance $\di$ between the $\theta$-barcodes of two $\R$-spaces $X$ and $Y$ is bounded above by the homotopy interleaving distance $\dhi$ between $X$ and $Y$ (see Definition \ref{def:dhi}).
%Furthermore, when the $\R$-diagrams are given by the Vietoris--Rips filtrations of two metric spaces, this upper bound can be replaced with twice the Gromov--Hausdorff distance $\dgh$ between the metric spaces. See Theorem \ref{thm:theta stability} for details.

% Let $X$ be a q-tame $\R$-space.
% Recall from \textsection \ref{ss:theta-barcodes} that any cohomology operation $\theta \in \cO(\kk,n; \kk,m)$ defines a morphism of persistence modules $\theta \colon \rH^n(X; \k) \to \rH^m(X; \k)$ for any $\R$-space.

%\subsubsection{}\label{lem:di stability}
%
%\lemma Given two $\R$-spaces $X$ and $Y$, we have
%\[
%\di(\thetamodule{X}, \thetamodule{Y}) \leq \di(X,Y).
%\]
%
%\begin{proof}
%	This follows directly from the fact that $X \mapsto \thetamodule{X}$ defines a functor from the category of spaces to the category of vector spaces over $\k$.
%	\anibal{I think more should be said.}
%\end{proof}

\subsubsection{}\label{lem:w.h.e. preservance}

\lemma If two cellular $\R$-spaces $X$ and $X'$ are weakly equivalent (\cref{ss:R-spaces}), then, for any linear cohomology operation $\theta$, one has
\[
\begin{split}
	\di(\img_\theta(X), \img_\theta(X')) &= 0, \\
	\di(\ker_\theta(X), \ker_\theta(X')) &= 0.
\end{split}
\]

\ling{Do we need the cellular structure? I see the reason of having it for $\Sq^k$ when the cohomology ring structure needs to be preserved.
For $\Sq^k_\ell$ and general $\theta\in \cO(\ell,m)$, we do not need the cohomology operation to preserve the ring structure (it only needs to preserved cohomology vector spaces.)}

\begin{proof}
	We write the proof only for $\img_\theta(X)$ since that for $\ker_\theta(X)$ is analogous.
	Given that $X$ and $X'$ are weakly equivalent, there exists an $\R$-space $Z$ and morphisms $f \colon Z \to X$ and $g \colon Z \to Y$ such that $f_r$ and $g_r$ are weak homotopy equivalences for any $r \in \R$.
	Without loss of generality we can assume $Z$ to be cellular using the cofibrant replacement functor of the model category structure on $\R$-spaces described in \cite{blumberg2023interleaving}.
	Both $f_r$ and $g_r$ induce an isomorphism on cohomology for each $r \in \R$, and the naturality of $\theta$ implies that $\img_\theta(Z)$ and $\img_\theta(X)$ are isomorphic, as well as $\img_\theta(Z)$ and $\img_\theta(X')$.
	Therefore,
	\[
	\di(\img_\theta(X),\img_\theta(X')) \leq
	\di(\img_\theta(X),\img_\theta(Z)) + \di(\img_\theta(Z),\img_\theta(X')) =
	0.\qedhere
	\]
	%	\ling{to continue: here we need to check that the image of $\theta$ is preserved up to isomorphism under w.h.e.}
	%	\anibal{A bit tricky... One know that a weak equivalence induces iso. in cohomology (Hatcher 4.28), but I am not sure about how to make the inverse consistent across $\R$. This issue should have already appeared the study of persistent homology and the homotopy interleaving distance.}
\end{proof}

\subsubsection{}\label{thm:theta stability}

\theorem
For cellular $\R$-spaces $X$ and $Y$ and a linear cohomology operation $\theta$ one has
\begin{align*}
	\di(\img_\theta(X), \img_\theta(Y)) &\leq \dhi(X,Y), \\
	\di(\ker_\theta(X), \ker_\theta(Y)) &\leq \dhi(X,Y).
\end{align*}

\begin{proof}
	We only write the proof for $\img_\theta(X)$ since that for $\ker_\theta(X)$ is similar.
	Take any $\delta > \dhi(X,Y)$.
	By the definition of the homotopy interleaving distance, there exist $\R$-spaces $X' \simeq X$ and $Y' \simeq Y$ such that $X'$ and $Y'$ are $\delta$-interleaved.
	Applying the cofibrant replacement functor $\rQ$ we get $d_\rI(\rQ X', \rQ Y') \leq \delta$ since applying any functor does not increase the interleaving distance.
	Both $\rQ X$ and $\rQ Y$ can be assumed to be cellular $\R$-spaces, since cofibrant objects in $\Top^\R$ are in particular objectwise cofibrant.
	By applying the triangle inequality and Lemma \ref{lem:w.h.e. preservance}, we obtain
	%	\begin{align*}
		%		\di(\thetamodule{X}, \thetamodule{Y}) \leq& \,
		%		\di(\thetamodule{X}, \thetamodule{X'}) + \di(\thetamodule{X'}, \thetamodule{Y'}) + \di(\thetamodule{Y'}, \thetamodule{Y}) \\ =& \,
		%		0 + \di(\thetamodule{X'}, \thetamodule{Y'}) + 0 \\ \leq \,&
		%		\di(X',Y') \\ \leq \,&
		%		\delta.
		%	\end{align*}
	\begin{align*}
		\di(\img_\theta(X), \img_\theta(Y)) \leq& \,
		\di(\img_\theta(X), \img_\theta(\rQ X')) \\ +& \,
		\di(\img_\theta(\rQ X'), \img_\theta(\rQ Y')) + \di(\img_\theta(\rQ Y'), \img_\theta(Y)) \\ =& \,
		0 + \di(\img_\theta(\rQ X'), \img_\theta(\rQ Y')) + 0 \\ \leq \,&
		\di(\rQ X', \rQ Y') \\ \leq \,&
		\delta.
	\end{align*}
	where, for the second inequality, we used Lemma~\ref{ss:interleaving}.
	Since $\delta > \dhi(X,Y)$ is arbitrary, we obtain the desired inequality.
\end{proof}

\subsubsection{}\label{cor:theta stability VR}

For a metric space \(\cX\) and a linear cohomology operation \(\theta\) we denote \(\img_\theta\big(\VR(\cX)\big)\) and \(\ker_\theta\big(\VR(\cX)\big)\) respectively by \(\img_\theta^\VR(\cX)\) and \(\ker_\theta^\VR(\cX)\).

\medskip\corollary For any two metric spaces $\cX$ and $\cY$ and linear cohomology operation~$\theta$,
\begin{align*}
	\di(\img_\theta^\VR(\cX),\, \img_\theta^\VR(\cY)) \leq 2 \cdot \dgh(\cX,\cY), \\
	\di(\ker_\theta^\VR(\cX),\, \ker_\theta^\VR(\cY)) \leq 2 \cdot \dgh(\cX,\cY).
\end{align*}

\begin{proof}
	From \cref{thm:stability-HI} and the previous theorem we obtain
	\[
	\di(\img_\theta^\VR(\cX), \img_\theta^\VR(\cY)) \leq
	\dhi(\VR\cX, \VR\cY) \leq 2 \cdot \dgh(\cX,\cY).
	\]
	The same argument applies to $\ker_\theta^\VR(\cX)$.
\end{proof}

We remark that if the metric spaces are compact or, more generally, totally bounded, the above persistent modules are q-tame and the bottleneck distance of their barcodes is bounded by twice the Gromov--Hausdorff distance, since bottleneck and interleaving distances agree.