\subsection{Stability}\label{ss:stability}

We prove the stability of cohomology operations using the homotopy interleaving distance of \cite{blumberg2023interleaving}.

% \textcolor{red}{To prove the stability of persistent cohomology operations, we establish lemmas showing that weak equivalences between cellular $\R$-spaces preserve the images and kernels of linear cohomology operations. }

\subsubsection{}\label{lem:w.h.e. preservance}

\lemma If two cellular $\R$-spaces $X$ and $X'$ are weakly equivalent (\cref{ss:R-spaces}), then, for any linear cohomology operation $\theta$
\[
\begin{split}
	\img_\theta(X) & \cong \img_\theta(X'), \\
	\ker_\theta(X)& \cong \ker_\theta(X').
\end{split}
\]

\begin{proof}
	We write the proof only for $\img_\theta(X)$ since that for $\ker_\theta(X)$ is analogous.
	Given that $X$ and $X'$ are weakly equivalent, there exists an $\R$-space $Z$ and morphisms $f \colon Z \to X$ and $g \colon Z \to X'$ such that $f_r$ and $g_r$ are weak homotopy equivalences for any $r \in \R$.
	Without loss of generality we can assume $Z$ to be cellular using the cofibrant replacement functor of the model category structure on $\R$-spaces described in \cite{blumberg2023interleaving}.
	Both $f_r$ and $g_r$ induce an isomorphism on cohomology for each $r \in \R$, and the naturality of $\theta$ implies that $\img_\theta(Z)$ and $\img_\theta(X)$ are isomorphic, as well as $\img_\theta(Z)$ and $\img_\theta(X')$.
    This finishes the proof.
	% Therefore,
	% \[
	% \di(\img_\theta(X),\img_\theta(X')) \leq
	% \di(\img_\theta(X),\img_\theta(Z)) + \di(\img_\theta(Z),\img_\theta(X')) =
	% 0.\qedhere
	% \]
\end{proof}

\subsubsection{}\label{thm:stability theta}

\lemma
For cellular $\R$-spaces $X$ and $Y$ and a linear cohomology operation $\theta$
\begin{align*}
	\di(\img_\theta(X), \img_\theta(Y)) &\leq \dhi(X,Y), \\
	\di(\ker_\theta(X), \ker_\theta(Y)) &\leq \dhi(X,Y).
\end{align*}

\begin{proof}
	We only write the proof for $\img_\theta(X)$ since that for $\ker_\theta(X)$ is similar.
	Take any $\delta > \dhi(X,Y)$.
	By the definition of the homotopy interleaving distance, there exist $\R$-spaces $X' \simeq X$ and $Y' \simeq Y$ such that $X'$ and $Y'$ are $\delta$-interleaved.
	Applying the cofibrant replacement functor $\rQ$ we get $d_\rI(\rQ X', \rQ Y') \leq \delta$ since applying any functor does not increase the interleaving distance.
	Both $\rQ X$ and $\rQ Y$ can be assumed to be cellular $\R$-spaces, since cofibrant objects in $\Top^\R$ are in particular objectwise cofibrant.
	By applying the triangle inequality and Lemma \ref{lem:w.h.e. preservance}, we obtain
	\begin{align*}
		\di(\img_\theta(X), \img_\theta(Y)) \leq& \,
		\di(\img_\theta(X), \img_\theta(\rQ X')) \\ +& \,
		\di(\img_\theta(\rQ X'), \img_\theta(\rQ Y')) + \di(\img_\theta(\rQ Y'), \img_\theta(Y)) \\ =& \,
		0 + \di(\img_\theta(\rQ X'), \img_\theta(\rQ Y')) + 0 \\ \leq \,&
		\di(\rQ X', \rQ Y') \\ \leq \,&
		\delta.
	\end{align*}
	where, for the second inequality, we used Lemma~\ref{ss:interleaving}.
	Since $\delta > \dhi(X,Y)$ is arbitrary, we obtain the desired inequality.
\end{proof}

\subsubsection{}\label{thm:stability VR}

For a metric space \(\cX\) and a linear cohomology operation \(\theta\) we denote \(\img_\theta\big(\VR(\cX)\big)\) and \(\ker_\theta\big(\VR(\cX)\big)\) respectively by \(\img_\theta^\VR(\cX)\) and \(\ker_\theta^\VR(\cX)\).

\theorem
For any two metric spaces $\cX$ and $\cY$ and a linear cohomology operation~$\theta$
\begin{align*}
	\di(\img_\theta^\VR(\cX),\, \img_\theta^\VR(\cY)) \leq 2 \cdot \dgh(\cX,\cY), \\
	\di(\ker_\theta^\VR(\cX),\, \ker_\theta^\VR(\cY)) \leq 2 \cdot \dgh(\cX,\cY).
\end{align*}

\begin{proof}
	From \cref{thm:stability-HI} and the previous theorem we obtain
	\[
	\di\big(\img_\theta^\VR(\cX), \img_\theta^\VR(\cY)\big) \leq
	\dhi\big(\VR(\cX), \VR(\cY)\big) \leq 2 \cdot \dgh(\cX,\cY).
	\]
	The same argument applies to $\ker_\theta^\VR$.
\end{proof}

We remark that if the metric spaces are compact or, more generally, totally bounded, the above persistent modules are q-tame and the bottleneck distance of their barcodes is bounded by twice the Gromov--Hausdorff distance, since bottleneck and interleaving distances agree.