% !TEX root = ../stability.tex

% Write new commands below

\newcommand{\bbR}{\mathbb{R}}

\definecolor{darkdgmcolor}{rgb}{0.0, 0.0, 0.8}
\definecolor{darkgreen}{rgb}{0.0, 0.8, 0.0}
\definecolor{purple}{RGB}{153,50,204}
\definecolor{dgmcolor}{RGB}{255,20,147}
\definecolor{barccolor}{RGB}{20,147,255}
\definecolor{myred}{RGB}{227,26,28}

\newcommand{\ling}[1]{{ \textcolor{purple} {#1}}}

\newcommand{\facundo}[1]{{ \textcolor{blue} {#1}}}

\DeclareMathOperator{\VR}{VR}
\DeclareMathOperator{\opH}{H}
% \newcommand{\sqbarc}[1]{\Sq^{#1}\barc}

\DeclareMathOperator{\barc}{Bar}
\DeclareMathOperator{\img}{Im}

\newcommand{\Hpt}[4][\Ftwo]{\opH_{#2}(\VR_{#4}(#3);\,#1)}
\newcommand{\Hbarc}[3][\Ftwo]{\barc \opH_{#2}(\VR_\bullet(#3);\,#1)}
\newcommand{\sqbarc}[3][\Ftwo]{\barc\Sq^{#2}(\VR_\bullet(#3);\,#1)}
\newcommand{\fillrad}[1]{\operatorname{FillRad}(#1)}
\newcommand{\field}{\mathbb{F}}

\newcommand{\rp}{\mathbb{RP}}
\newcommand{\bbS}{\mathbb{S}} % for sphere
\newcommand{\diam}{\operatorname{diam}}

\newcommand{\db}{d_{\mathrm{B}}}
\newcommand{\dgh}{d_{\mathrm{GH}}}

\newcommand{\Coordinate}[2]%
{ \coordinate (#1) at (#2);
}