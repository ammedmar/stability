% !TEX root = ../vr_st.tex

\newcommand{\Rad}{\mathrm{Rad}}
\newcommand{\VS}{\mathbb{VS}}
\DeclareMathOperator{\sus}{\Sigma}
\renewcommand{\Vec}{\mathsf{Vec}}
\newcommand{\bbR}{\mathbb{R}}
\newcommand{\RR}{\mathbb{R}}
\newcommand{\kk}{\mathbb{k}}
\newcommand{\st}{\mathrm{st}}

\definecolor{darkdgmcolor}{rgb}{0.0, 0.0, 0.8}
\definecolor{darkgreen}{rgb}{0.0, 0.8, 0.0}
\definecolor{purple}{RGB}{153,50,204}
\definecolor{dgmcolor}{RGB}{255,20,147}
\definecolor{barccolor}{RGB}{20,147,255}
\definecolor{myred}{RGB}{227,26,28}

\definecolor{darkblue}{rgb}{0,0,0.7} % darkblue color
\newcommand{\darkblue}{\color{darkblue}} % darkblue command
\newcommand{\defn}[1]{{\darkblue \emph{#1}}} % for new concepts


\DeclareMathOperator{\rad}{rad}
\newcommand{\VR}{\mathrm{VR}}
\DeclareMathOperator{\opH}{H}
\newcommand{\PFD}{\mathrm{PFD}}
\newcommand{\Sq}{\mathrm{Sq}}
% \newcommand{\sqbarc}[1]{\Sq^{#1}\barc}

\renewcommand{\corollary}{\medskip\noindent\textit{\darkblue Corollary}.\ }
\renewcommand{\theorem}{\medskip\noindent\textit{\darkblue Theorem}.\ }
\renewcommand{\lemma}{\medskip\noindent\textit{\darkblue Lemma}.\ }
\renewcommand{\proposition}{\medskip\noindent\textit{\darkblue Proposition}.\ }

\theoremstyle{theorem}
\newtheorem{introtheorem}{Theorem}
\renewcommand{\theintrotheorem}{\Alph{introtheorem}}

\DeclareMathOperator{\barc}{Bar}
\DeclareMathOperator{\img}{Im}
\renewcommand{\ker}{\operatorname{Ker}}

\newcommand{\field}{\k}

\newcommand{\bbS}{\mathbb{S}}
\newcommand{\rp}{\mathbb{RP}}
\newcommand{\cp}{\mathbb{CP}}
\newcommand{\diam}{\operatorname{diam}}

\newcommand{\db}{d_{\mathrm{B}}}
\newcommand{\dgh}{d_{\mathrm{GH}}}

\newcommand{\Coordinate}[2]%
{ \coordinate (#1) at (#2);
}

% Added by Ling
\newcommand{\thetabarc}[1]{\barc\img_{\theta}^{\VR} (#1)}
\newcommand{\thetamodule}[1]{\img\theta_{#1}}
\newcommand{\dhi}{d_{\mathrm{HI}}}
\newcommand{\di}{d_{\mathrm{I}}}
\newcommand{\Xfunc}{X_\bullet}
\newcommand{\Yfunc}{Y_\bullet}
\newcommand{\Wfunc}{W_\bullet}
\newcommand{\cost}{\mathrm{cost}}
\newcommand{\Hpt}[4][\Ftwo]{\opH_{#2}(\VR_{#4}(#3);\,#1)}
\newcommand{\Hbarc}[3][\Ftwo]{\barc \opH_{#2}^{\VR} (#3)}
\newcommand{\sqbarcl}[3]{\barc\img_{\Sq^{#1}_{#2}}^{\VR} (#3)}

\newcommand{\fillradnofield}[1]{\operatorname{FillRad}(#1)}
\newcommand{\sqbarcgeneral}[2]{\barc\Sq^{#1}(\VR #2)}
\newcommand{\ling}[1]{\noindent{\textcolor{purple}{\underline{Ling}: #1}}}
\newcommand{\lz}[1]{\todo[\textcolor{purple}{Ling:}\!]{{#1}}}

% Commands for critical values
\newcommand{\fillrad}{\mathrm{Rad}_{\mathrm{fill}}}
\newcommand{\crit}{\mathrm{Rad}_{\mathrm{htpy}}}
\newcommand{\firstdeath}[2]{\mathrm{Rad}_{#1}(#2)}
